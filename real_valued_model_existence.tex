%------------------------------------------------------------------------------
% Beginning of journal.tex
%------------------------------------------------------------------------------
%
% AMS-LaTeX version 2 sample file for journals, based on amsart.cls.
%
%        ***     DO NOT USE THIS FILE AS A STARTER.      ***
%        ***  USE THE JOURNAL-SPECIFIC *.TEMPLATE FILE.  ***
%
% Replace amsart by the documentclass for the target journal, e.g., tran-l.
%
\documentclass{amsart}

\usepackage{graphicx}
\usepackage{enumitem}
\usepackage[mathscr]{euscript}
\usepackage{amssymb}
\usepackage{hyperref}
\usepackage{proof}
\usepackage{cite}
\usepackage[toc,page]{appendix}

\newtheorem{theorem}{Theorem}[section]
\newtheorem{lemma}[theorem]{Lemma}
\newtheorem{corollary}[theorem]{Corollary}

\theoremstyle{definition}
\newtheorem{remark}[theorem]{Remark}
\newtheorem{definition}[theorem]{Definition}
\newtheorem{notation}[theorem]{Notation}
\newtheorem{example}[theorem]{Example}

\renewcommand{\descriptionlabel}[1]{\hspace\labelsep\normalfont\itshape #1:}

\numberwithin{equation}{theorem}

\renewcommand{\phi}{\varphi}
\renewcommand{\epsilon}{\varepsilon}

\newcommand{\R}{\mathbb{R}}
\newcommand{\Q}{\mathbb{Q}}
\newcommand{\N}{\mathbb{N}}
\newcommand{\V}{\mathbf}
\newcommand{\restrict}[2]{{#1\restriction#2}}
\newcommand{\powerset}[1]{\wp(#1)}
\newcommand{\where}{\mid}
\newcommand{\unvee}{{\vee}}
\newcommand{\unwedge}{{\wedge}}
\newcommand{\proves}{\vdash}

\newcommand{\strict}[1]{{\left\lfloor#1\right\rfloor}}
\newcommand{\lrarr}{\leftrightarrow}
\newcommand{\rat}[1]{{\overline{#1}}}
\newcommand{\num}[1]{{\bar{\bar{#1}}}}
%\newcommand{\narrow}[1]{\to^{(#1)}}
\newcommand{\narrow}[1]{\xrightarrow{#1}}
\renewcommand{\to}{\narrow{}}
\newcommand{\len}{\ell}

\newcommand{\arr}{{\to}}
\newcommand{\intro}{\!\operatorname{I}}
\newcommand{\elim}{\!\operatorname{E}}
\newcommand{\trans}{\text{\sc{Trans}}}
\newcommand{\aref}[1]{\!\text{\ref{itm:axiom-#1}}}
\newcommand{\ordef}{\unvee\text{\sc{Def}}}
\newcommand{\swap}{\text{\sc{Swap}}}
\newcommand{\pushdown}{\arr\text{\sc{Manip}}}
\newcommand{\generalization}{\text{\sc{Gen}}}

\newcommand{\baselang}{\operatorname{L}}
\newcommand{\lang}{\baselang_{\omega_1\omega}}
\newcommand{\Tskolem}{{T_{\operatorname{Sk}}}}
\newcommand{\sig}{{\operatorname{S}}}
\newcommand{\sigsk}{\sig_{\operatorname{Sk}}}
\newcommand{\frag}{\mathcal{L}}
\newcommand{\fragsk}{\frag_{\operatorname{Sk}}}
\newcommand{\skolem}[2]{F_{{#1},{#2}}}
\newcommand{\hull}{\operatorname{Hull}}

\newcommand{\luk}{{\L}ukasiewicz}


\newenvironment{claim}[1]{\vspace{.07in}\par\noindent\emph{Claim #1.}\space}{}
\newenvironment{claimproof}{\vspace{.07in}\par\noindent\emph{Proof of Claim.}\space}{\hfill \emph{End of Claim.}}

\begin{document}

\title{Model Existence for $[0,1]$-Valued Logic}

\author{Joe Grinstead}

%    Address of record for the research reported here
\address{Department of Mathematics, CMU}
%    Current address
\curraddr{Department of Mathematics, CMU}

\email{jgrinste@andrew.cmu.edu}
%    \thanks will become a 1st page footnote.
%\thanks{The first author was supported in part by NSF Grant \#000000.}

%    Information for second author
%\author{Author Two}
%\address{Mathematical Research Section, School of Mathematical Sciences,
%Australian National University, Canberra ACT 2601, Australia}
%\email{two@maths.univ.edu.au}
%\thanks{Support information for the second author.}

%    General info
%\subjclass[2000]{Primary 54C40, 14E20; Secondary 46E25, 20C20}

\date{\today}

%\dedicatory{This paper is dedicated to our advisors.}

%\keywords{Differential geometry, algebraic geometry}

\begin{abstract}
	We explore a $[0,1]$-valued analogue of the logic $\lang$.
	The core result of this paper is an adaptation of the model existence theorem to this logic.
	From there we prove versions of completeness, establish the Hanf number to be $\beth_{\omega_1}$.
	Finally, we show how the work relates to fuzzy logic and continuous logic.
\end{abstract}

\maketitle
\tableofcontents
\newpage

\section*{Introduction}
In this paper, we study an infinitary logic that takes truth values in the closed interval $[0,1]$, where 1 behaves as truth, 0 as falsehood, and the values in between as partial truths.
The logic is analogous to $\lang$ from classical logic.

The genesis of $[0,1]$-valued logics, or any multi-valued logic, can be traced back to early last century with the work of Polish logician Jan {\luk}.
In a published address to Warsaw University~\cite{lukasiewicz1918farewell}, {\luk} reflected on his philosophically-motivated three-valued logic in which sentences could be true, false, or possible.
By 1930, he and Tarski had expanded the possible truth-values to all of $[0,1]$.
Completeness was proven algebraically by Chang~\cite{chang1958algebraic,chang1959new} using a new notion of MV-algebras (standing for ``multi-valued'') as the semantics of the logic.

The late 1960s saw the advent of \emph{fuzzy logics}.
Though the term ``fuzzy'' was coined by Zadeh~\cite{zadeh1965fuzzy} to describe sets where membership is a value from 0 to 1 instead of simply true or false, Goguen~\cite{goguen1969logic} was the first to create a fuzzy logic (dubbed the ``logic of inexact concepts'').
Fuzzy logics are based on formal and precise deduction, deriving its name from the \emph{statements} being fuzzy, not the logic itself.
For a thorough history, introduction, and examination of the field, read ``Metamathematics of Fuzzy Logics''~\cite{hajek1998metamathematics}.

The syntax of fuzzy logics is taken from standard logic, meaning there are formulas built from relations, connectives, and quantifiers.
The main connectives are:
\begin{itemize}
\item The symbol for false: $\rat 0$ or $\bot$.
\item The symbol for implication: $\to$.
\item The symbols for conjunction: $\wedge$ (arithmetic) and $\odot$ (multiplicative)\footnote{Semantically, arithmetic conjunction is the infimum of its constituents whereas multiplicative conjunction is defined per logic.}
\end{itemize} 
Fuzzy logicians have experimented with various interpretations and properties of these connectives, making for an abundance of fuzzy logics (18 are given in~\cite{metcalfe2008proof}); however, 
there are three logics considered particularly important: G\"odel, {\luk}, and Product.
This paper focuses on the interpretations in {\luk} logic with truth-values in $[0,1]$, the most important being that for `$\to$' (called the \emph{{\luk} implication}).

Regular completeness does not hold in {\luk} logic except when restricted to finite theories~\cite{hajek1997strong}.
Jan Pavelka wrote a series of articles~\cite{pavelka1979fuzzy} in which he attained an approximate form of completeness after adding a truth-constant (ie.\ nullary connective) for each number in $[0,1]$.
H\'ajek refers to Pavelka's completeness result as \emph{completeness} (relegating standard completeness to \emph{strong completeness}), whereas others sometimes say \emph{Pavelka-style completeness}; we opt for Pavelka-style completeness in this paper.
The language being necessarily uncountable is non-ideal. Fortunately, though, Pavelka-style completeness holds for {\luk} logic extended with a connective only for each rational\footnote{Interestingly, but not of use to this paper, for every sentence $\phi$ in {\luk} logic and rational $q\in\Q\cap[0,1]$, there is a {\luk} sentence $\psi$ which evaluates to 1 exactly in those structures where $\phi$ evaluates to $q$~(see Section~3.3~of~\cite{hajek1998metamathematics}).
This means that many properties of Pavelka's logic can be derived from that of {\luk}'s.}, rather than the whole unit interval~\cite{hajek1995fuzzy}.
For this paper, we refer to predicate {\luk} logic extended with rational connectives as \emph{{\luk}-Pavelka logic}.

The logic investigated in this paper is {\luk}-Pavelka logic extended with countable arithmetic conjunction\footnote{Because we use arithmetic conjunction much more than multiplicative, we will often just write ``conjunction'' instead of ``arithmetic conjunction.''}.
This logic most closely resembles that defined in~\cite{eagle2014omitting}, though we do not include a symbol, fuzzy or otherwise, for equality. We refer to the logic as \emph{infinitary $[0,1]$-valued logic}.
This paper adapts chapters 3, 4, and 15 of~\cite{keisler1971model} to infinitary $[0,1]$-valued logic.
The key result is an adaptation of the \emph{model existence theorem} that allows it to be used similarly to the classical setup.
We define \emph{fragments} similarly to classical $\lang$ and prove completeness both in general and for certain fragments in the same way as~\cite{keisler1971model}, which used a Henkin construction formalized by Makai.
For general fragments, we introduce additional inference rules that give Pavelka-style completeness and/or completeness.
The remaining original material is a proof that infinitary $[0,1]$-valued logic has Hanf number $\beth_{\omega_1}$.

In this paper, we introduce a particularly important conjunction schema called the \emph{floor formula}.
Interestingly, one can show that the floor formula is equivalent to the \emph{Baaz delta}, $\triangle$, a unary connective\footnote{For each formula $\phi$ and structure $\mathcal M$, we define $[\triangle\phi]^\mathcal M:=\left\{\begin{array}{ll}
    1&\text{if $\phi^\mathcal M=1$ }\\
    0&\text{otherwise}
  \end{array}\right.$}
introduced by~\cite{baaz1996infinite} to give a completeness result for G\"odel logic when extended with some additional inference rules.
In H\'ajek's book~\cite{hajek1998metamathematics}, he shows that the Baaz delta and its associated inference rules are useful for {\luk} and Product logic as well.
This paper demonstrates that the additional axioms and inference rules are redundant in infinitary $[0,1]$-valued logic.

Baaz (with Metcalfe) also showed that extending {\luk} logic with a particular infinitary inference rule gives completeness~\cite{baaz2007lukasiewicz}.
While the inference rule is not used in this paper, in Section~\ref{sec:fuzzy-logic} we discuss how the work in this paper presents an interesting perspective on the new rule.

Though confined to a single section, this investigation originally began as a study of \emph{first-order continuous logic}.
The field can be traced back to Chang and Kiesler's ``Continuous Model Theory''~\cite{chang1966continuous}, but was relatively quiet for decades.
This century has marked a strong interest in the area, particularly after Ben Yaacov and Usvyatsov in~\cite{ben2010continuous} showed Henson's logic for Banach spaces\footnote{See~\cite{henson2002ultraproducts} more information about Henson's logic.} could be reformulated towards their logic, a less general variant of Chang and Kiesler's that constitutes the modern first-order continuous logic.
For a full, self-contained text on the subject, see~\cite{yaacov2008model}.

First-order continuous logic is similar to fuzzy logic, except that each function and relation symbol has an associated \emph{modulus of continuity} and structures must satisfy each modulus with respect to a bundled metric.
The metric is treated as a binary relation in the logic, with some axiom schema for each modulus of continuity~\cite{yaacov2010proof}.
A metric indirectly appears in predicate fuzzy logic as well, in that there is a symbol for \emph{similarity} `$\approx$' with $(a,b)\mapsto 1-(a\approx b)$ being a metric.
If similarity is included in a logic, then axioms are added stipulating that each function and relation is 1-Lipschitz.
As stated before in this introduction, we do not include any metric or similarity in our logic; however, all the axiom schema are countable when restricted to a countable fragment, so we obtain results for the two logics above by simply adding the axioms to the theory.
This approach gives us structures satisfying the metric as a \emph{pseudometric}, but this is a standard situation (called a \emph{prestructure} in~\cite{yaacov2008model}) and simply taking equivalence classes of elements gives a proper structure.


Although in~\cite{ben2010continuous} they allow any continuous function to be a connective, Caicedo and Iovino~\cite{caicedo2012omitting} showed that the {\luk} implication with rational truth-constants was just as expressive.

Chris Eagle~\cite{eagle2014omitting} demonstrated omitting types for infinitary continuous first-order logic.
Omitting types is particularly tricky for continuous first-order logic rather than first-order {\luk}-Pavelka logic because structures are required to be complete with respect to their metric in continuous first-order.
In finite continuous logic, every formula is necessarily continuous, so taking a completion gives an elementary super-structure.
Unfortunately, not every formula must be continuous in infinitary $[0,1]$-valued logic.
To counter this, Eagle introduced \emph{continuous fragments} for which his results hold.
We use the same concept in this paper when referring to continuous logic, though our definition of fragment includes an additional requirement needed for the completeness result.

This paper is self-contained with respect to fuzzy and continuous logic; however, we assume the reader has seen some logic and model theory before.
Notationally, we have chosen to err on the side of fuzzy logicians, as in \cite{hajek1998metamathematics} and \cite{metcalfe2008proof}, rather than the notation used in \cite{yaacov2008model} because the fuzzy logic notation makes the connections to standard logic more clear.

\newpage
\section{Definitions, Concepts, and Notation}
\begin{definition}
  A \emph{signature} consists of a set of function symbols (each with an associated finite arity), a set of relation symbols (again with associated finite arity), and a set of constant symbols.
  A \emph{countable signature} contains only countably many symbols.
  The notion of \emph{term}, \emph{atomic formula}, and \emph{free-variable} are the same as in classical first-order logic.
  A \emph{closed term} is a term with no free-variables.
\end{definition}
\begin{definition}
  For a given signature $S$, we define the set of formulas $\lang(S)$ recursively as:
  \begin{itemize}
  \item Every atomic formula of $S$ is in $\lang(S)$.
  \item For each $q\in\Q\cap[0,1]$, the formula $\rat q$ is in $\lang(S)$.
  \item For each $\phi$ and $\psi$ in $\lang(S)$, the formula $\phi\to\psi$ is in $\lang(S)$.
  \item For each countable set of formulas $\Phi$ in $\lang(S)$, if the formulas in $\Phi$ together reference only finitely many free-variables, then the formula $\unwedge\Phi$ and $\unvee\Phi$ is in $\lang(S)$.
  \item For each $\phi$ in $\lang(S)$, the formulas $\forall x\,\phi$ and $\exists x\,\phi$ are in $\lang(S)$.
  \end{itemize}
  Of course, we let $x$ be any of countably many free-variable symbols in that last rule.
  Additionally, the set of formulas $\baselang(S)$ is defined in the same way, except that conjunction and disjunction are restricted to pairs of formulas instead of countable sets.
  A \emph{sentence} is a formula with no free-variables.
\end{definition}
\begin{notation}
  When the signature is unambiguous, we write $\baselang$ and $\lang$ instead of $\baselang(S)$ and $\lang(S)$, respectively.
  Many times in this paper, the signature does not matter and so a result or definition will be stated without reference to a signature; in those cases, one should understand there to be an implicitly fixed and arbitrary signature.
\end{notation}
\begin{notation}
  For any rational, we opt to distinguish the number itself from its corresponding nullary relation by putting a bar over the rational (eg. $\rat q$ instead of $q$ or $\rat 0$ instead of $0$).
  This notation is useful and is utilized throughout~\cite{hajek1998metamathematics}; however, it prevents us from using the bar to mean a tuple.
  Our solution is to use bold-face variables (eg. $\V a$ and $\V x$) for finite tuples (we never have infinite tuples in this paper).
\end{notation}
\begin{notation}
  For any tuple $\V x$, we take $\len(\V x)$ to be the length of $\V x$.
\end{notation}
At this point, we have defined the syntax of our logic, but have yet to define its semantics.
The following definition is how we interpret the arrow connective.
While at first it may seem unnatural, it is fundamentally powerful, as demonstrated later in Theorem~\ref{thm:expressive-logic}.
\begin{definition}
  The \emph{{\L}ukasiewicz implication} is the function `$\Rightarrow$'$:[0,1]^2\to[0,1]$, written in infix notation, defined for all $x,y\in[0,1]$ as
  \[
    (x\Rightarrow y) := \min\{1-x+y,\ 1\}.
  \]
\end{definition}
\begin{remark}
  For any $x,y\in[0,1]$, the value of $x\Rightarrow y$ is 1 if and only if $x\leq y$.
  In particular, for all $x\in[0,1]$, both $0\Rightarrow x$ and $x\Rightarrow 1$ are 1.
\end{remark}
\begin{definition}
  For any signature $S$, the definition of an $S$-structure is the same as for classical first-order logic except that the codomain of relations is $[0,1]$ instead of $\{0,1\}$.
  We will use script letters for structures (eg. $\mathcal M$ and $\mathcal N$), roman letters for their elements (eg. $M$ and $N$), and superscripts to indicate the interpretation of functions and relations (eg. $R^\mathcal M$ for the interpretation of $R$ in $\mathcal M$).
\end{definition}
\begin{definition}
  Let $S$ be a signature and $\mathcal M$ a $S$-structure.
  For each formula $\phi(\V x)$ in $\lang(S)$ and tuple $\V a\in M^{\len(\V x)}$ (meaning $\V a$ is a tuple of elements in $\mathcal M$ that has the same length as $\V x$), define the \emph{truth-value} of $\phi[\V a]$ in $\mathcal M$, written $\phi^\mathcal M(\V a)$, recursively as:
  \begin{itemize}
  \item If $\phi(\V x)$ is of the form $R(\V x)$, where $R$ is a relation, then $\phi^\mathcal M(\V a):=R^\mathcal M(\V a)$.
  \item If $\phi(\V x)$ is of the form $\rat p$, where $p\in\Q\cap[0,1]$, then $\phi^\mathcal M(\V a):=p$.
  \item If $\phi(\V x)$ is of the form $\psi(\V x)\to\chi(\V x)$, then $\phi^\mathcal M(\V a):=(\psi^\mathcal M(\V a)\Rightarrow\chi^\mathcal M(\V a))$.
  \item If $\phi(\V x)$ is of the form $\bigwedge_{\psi\in\Psi}\psi(\V x)$, then $\phi^\mathcal M(\V a):=\inf\{\psi^\mathcal M(\V a)\where \psi\in\Psi\}$.
  \item If $\phi(\V x)$ is of the form $\bigvee_{\psi\in\Psi}\psi(\V x)$, then $\phi^\mathcal M(\V a):=\sup\{\psi^\mathcal M(\V a)\where \psi\in\Psi\}$.
  \item If $\phi(\V x)$ is of the form $\forall y\,\psi(\V x;y)$, then $\phi^\mathcal M(\V a):=\inf\{\psi^\mathcal M(\V a;b)\where b\in M\}$
  \item If $\phi(\V x)$ is of the form $\exists y\,\psi(\V x;y)$, then $\phi^\mathcal M(\V a):=\sup\{\psi^\mathcal M(\V a;b)\where b\in M\}$
  \end{itemize}
  In addition, for any formula $\phi$ with $n$ free-variables for some $n<\omega$, we define
  \[
    \phi^\mathcal M:=\inf\{\phi^\mathcal M(\V a)\where \V a\in M^n\}.
  \]
\end{definition}
\begin{definition}
  For any structure $\mathcal M$ and formula $\phi$, we say $\mathcal M\models\phi$ iff $\phi^\mathcal M=1$.
  This means for that $\mathcal M\models\phi\to\psi$ for formulas $\phi$ and $\psi$ iff $\phi^\mathcal M\leq\psi^\mathcal M$.
\end{definition}
\begin{remark}\label{rem:issues}
  The semantic definition of $\unwedge$, $\unvee$, $\forall$, and $\exists$ as infs and sups should be relatively intuitive.
  Unfortunately, these definitions can (and do) create difficulties, two in particular:
  \begin{itemize}
  \item 
    It's possible for $\bigvee_{\psi\in\Psi}\psi$ to be true (ie. evaluates to 1) without any of $\psi\in\Psi$ being true.
    The same problem exists for $\exists x\,\psi(x)$, as it may be true without being witnessed.
  \item
    Proof-theoretically, the $\wedge$ connective does not coordinate too well with the $\to$ connective; consider that the formula $(\phi\wedge\psi)\to\chi$ is not necessarily the same as $\phi\to(\psi\to\chi)$ and that $\phi\to\psi$ is not expressible with the $\wedge$ and $\neg$ (defined in Notation~\ref{not:shorthand}) symbols.
  \end{itemize}
  The first problem is tackled by observing that if $\unwedge\Psi$ is true, then for every rational $p\in\Q\cap[0,1]$ that is \emph{strictly} below 1, there is a formula $\psi$ that has a truth-value greater than $p$.
  The second problem is not an issue in fuzzy logic, as fuzzy logicians use an additional type of conjunction called a ``t-norm'' that works well with the $\to$ connective.
  The t-norm, though, is not a truly natural concept when doing model theory, so we opt not to include it in this work.
\end{remark}
\begin{notation}\label{not:shorthand}
  Throughout this paper, we use the following shorthand for any formulas $\phi$ and $\psi$:
  \begin{itemize}
  \item $\neg\phi$ represents $\phi\to\rat 0$.
  \item $\phi\wedge\psi$ represents $\unwedge\{\phi,\psi\}$.
  \item $\phi\vee\psi$ represents $\unvee\{\phi,\psi\}$.
  \item $\phi\leftrightarrow\psi$ represents $(\phi\to\psi)\wedge(\psi\to\phi)$.
  \end{itemize}
\end{notation}
\begin{remark}
  Finite conjunction and disjunction can be expressed with the $\to$ connective, so many authors take finite conjunction and disjunction as shorthand as well, though we do not do so here.
\end{remark}
\begin{remark}
  For any formulas $\phi$ and $\psi$ and structure $\mathcal M$, the truth-value of $\neg\phi$ and $\phi\leftrightarrow\psi$ are
  \[
    [\neg\phi]^\mathcal M=1-\phi^\mathcal M\quad\quad\text{and}\quad\quad [\phi\leftrightarrow\psi]^\mathcal M=1-|\phi^\mathcal M-\psi^\mathcal M|.
  \]
\end{remark}
\begin{definition}
  For any formulas $\phi$ and $\psi$, we take $\phi\equiv\psi$ to mean they have the same truth-value in all structures.
\end{definition}
\begin{remark}
Infinitary logics need ways to manipulate and argue about their infinitary sentences.
  In classical logic, Demorgan's laws give that for any countable set of formulas $\Phi$:
  \begin{align*}
    \neg\bigvee_{\phi\in\Phi}\phi &\equiv \bigwedge_{\phi\in\Phi}\neg\phi&
    &\text{and}&
    \neg\bigwedge_{\phi\in\Phi}\phi &\equiv \bigvee_{\phi\in\Phi}\neg\phi
  \end{align*}
  and while we're at it, we observe that for any formula $\phi(x)$:
  \begin{align*}
    \neg\exists x\,\phi(x) &\equiv \forall x\,\neg\phi(x)&
    &\text{and}&
    \neg\forall x\,\phi(x) &\equiv \exists x\,\neg\phi(x)
  \end{align*}
  These manipulations were found to be fundamental to studying classical $\lang$.
  Unfortunately, as brought up in Remark~\ref{rem:issues}, the $\neg$ symbol is not very powerful in our logic.
  Instead, we must focus our attention on the $\to$ symbol, which has a similar set of equivalences (where $\psi$ is a formula and $\Phi$ is a countable set of formulas):
  \begin{align}\label{eqn:arrow-manip}
    \Big(\bigvee_{\phi\in\Phi}\phi\Big)\to\psi&\equiv \bigwedge_{\phi\in\Phi}(\phi\to\psi)\\
    \Big(\bigwedge_{\phi\in\Phi}\phi\Big)\to\psi&\equiv \bigvee_{\phi\in\Phi}(\phi\to\psi)\\
    \psi\to\Big(\bigvee_{\phi\in\Phi}\phi\Big)&\equiv \bigvee_{\phi\in\Phi}(\psi\to\phi)\\
    \psi\to\Big(\bigwedge_{\phi\in\Phi}\phi\Big)&\equiv \bigwedge_{\phi\in\Phi}(\psi\to\phi)
  \end{align}
  and (where $\phi(x)$ and $\psi$ are formulas such that $x$ is not free in $\psi$):
  \begin{align}
    (\exists x\,\phi(x))\to\psi &\equiv \forall x\,(\phi(x)\to\psi)\\
    (\forall x\,\phi(x))\to\psi &\equiv \exists x\,(\phi(x)\to\psi)\\
    \psi\to(\exists x\,\phi(x)) &\equiv \exists x\,(\psi\to\phi(x))\\
    \psi\to(\forall x\,\phi(x)) &\equiv \forall x\,(\psi\to\phi(x))\label{eqn:arrow-manip-end}
  \end{align}
\end{remark}
The labeled equivalences above give us an important definition that is original to this paper: \emph{arrow-manipulations}.
\begin{definition}\label{def:arrow-manipulation}
  Let $\phi$ and $\psi$ be formulas.
  We say that $\psi$ is an \emph{arrow-manipulation} of $\phi$ (and $\phi$ an arrow-manipulation of $\psi$) if $\phi\equiv\psi$ is one of the equivalences described by \eqref{eqn:arrow-manip} through \eqref{eqn:arrow-manip-end}.
  For example, if $\phi$ is of the form $\forall x(\chi_1\to\chi_2(x))$ and $\psi$ is of the form $\chi_1\to(\forall x\,\chi_2(x))$, then $\phi$ and $\psi$ are arrow-manipulations of each other because of equivalence~\eqref{eqn:arrow-manip-end}
\end{definition}
\begin{remark}
  The notion of an arrow-manipulation is critical to the completeness section, which necessitates adding them to our notion of fragment (which is designed to serve the same purpose as the notion of fragment for $\lang$), defined below this remark.
  In this way, we differ from~\cite{eagle2014omitting}, which did not include arrow-manipulations in the fragment.
\end{remark}
  \begin{definition}\label{def:fragment}
    A set of formulas $\mathcal L$ is called a \emph{fragment} of $L_{\omega_1,\omega}$ if all the following hold:
    \begin{itemize}
    \item $\mathcal L$ contains all atomic formulas and rational connectives.
    \item $\mathcal L$ is closed under substituting terms and variables.
    \item $\mathcal L$ is closed under $\rightarrow$, $\forall$, $\exists$, and finite $\wedge$ and $\vee$.
    \item If $\phi\in\mathcal L$, then all subformulas of $\phi$ are in $\mathcal L$.
    \item If $\phi\in\mathcal L$, then all arrow-manipulations of $\phi$ are in $\mathcal L$.
    \end{itemize}
  \end{definition}
\begin{remark}
  For any countable set of formulas, there is a countable fragment containing that set.
\end{remark}
\begin{remark}
  One of the most striking differences between classical and {\L}ukasiewicz logic is that for formulas $\phi$ and $\psi$, the following may hold:
  \[
    \phi\to\psi\not\equiv \phi\to(\phi\to\psi)
  \]
  Indeed, for any structure $\mathcal M$, we see by invoking definitions that
  \begin{align*}
    &&[\phi\to\psi]^\mathcal M&=\min\{1-\phi^\mathcal M+\psi^\mathcal M,\ 1\}\\
    \text{and}&&[\phi\to(\phi\to\psi)]^\mathcal M&=\min\{2(1-\phi^\mathcal M)+\psi^\mathcal M,\ 1\}.
  \end{align*}
  The concept of contraction \emph{failing} is not initially intuitive; however, contraction fails only if the hypothesis is not 1, so the intuition is that invoking a faulty assumption twice is more questionable than invoking it once.
  The lack of contraction is a real problem though, beyond the intuition issue, as it invalidates the classical deduction theorem.
  To accommodate, we are forced to add new notation that will be used heavily in and after the completeness section.
\end{remark}
\begin{notation}\label{not:strict}
  For any $n<\omega$ and formulas $\phi$ and $\psi$, we write $\phi\narrow n\psi$ to mean the formula:
  \[
    \underbrace{\phi\to\phi\to\cdots\to\phi\to\phi}_\text{$n$ times}\to\psi
  \]
  Note that for this definition we take `$\to$' to be right-associative, so that $\phi\narrow3\psi$ represents $\phi\to(\phi\to(\phi\to\psi))$.
  Note as well that $\phi\narrow 0\psi$ is just $\psi$.
  In addition, for any formula $\phi$, we write $\strict\phi$ to mean the formula:
  \[
    \bigwedge_{n<\omega}\neg(\phi\narrow n\rat 0)
  \]
\end{notation}
\begin{remark}
  For any $n<\omega$, structure $\mathcal M$, and formulas $\phi$ and $\psi$ with no shared free-variables, the truth-value of $\phi\narrow n\psi$ in $\mathcal M$ is $\min\{n(1-\phi^\mathcal M)+\psi^\mathcal M,\ 1\}$.
\end{remark}
\begin{remark}\label{rem:context-chain}
  Consider any structure $\mathcal M$.
  For any formula $\phi$, if $\phi^\mathcal M\neq 1$, then there is some $n<\omega$ such that $n(1-\phi^\mathcal M)\geq 1$, which implies $[\neg(\phi\narrow n\rat 0)]^\mathcal M=1$.
  This means that for any formula $\phi$,
  \[
    \strict{\phi}^\mathcal M=\lfloor\phi^\mathcal M\rfloor,
  \]
  which thus explains the chosen notation.
\end{remark}
We now end this section with a theorem that was promised before the definition of {\L}ukasiewicz implication, and is included here merely to demonstrate the expressiveness of the language.
Note that this is the only measure-theoretic idea in the paper and may be skipped, so we do not give background in this paper on the concepts used.
\begin{theorem}\label{thm:expressive-logic}
  Fix $n<\omega$ and let $S$ be a signature with nullary relations $R_1,\dots,R_n$.
  For every Borel-measurable function $f:[0,1]^n\to[0,1]$, there is an $\lang(S)$-sentence $\phi$ such that for all $\V x\in[0,1]^n$, $f(\V x)=\phi^\mathcal M$ in all models $\mathcal M$ satisfying $R^\mathcal M_i=x_i$ for $i\in\{1,\dots,n\}$.
  \begin{proof}
    For simplicity, we assume $n=1$ and write $R$ instead of $R_1$.
    
    For any $p,q\in\Q\cap[0,1]$ define the formula $\chi_{[p,q]}$ to be:
    \[
      \strict{(\rat p\to R)\wedge (R\to\rat q)}
    \]
    If we consider the truth-value $\chi_{[p,q]}$ to be a function of the truth-value of $R$, we see that $\chi_{[p,q]}$ is the indicator of $R$ being in $[p,q]$.
    From there, we may use Dynkin's $\Pi$-$\Lambda$ theorem to show that the set
    \[
      \{A\subset[0,1]\where \text{$A$ is Borel and there is an indicator sentence for $R$ being in $A$}\}
    \]
    is a $\sigma$-algebra and thus contains all Borel sets, so we may define $\chi_A$ for each Borel set $A$ to be the indicator sentence of $R$ being in $A$.
    
    For every $r\in[0,1]$, the formula $\bigwedge_{q\in\Q\cap[r,1]}\rat q$, which we denote $\num r$, will always have truth-value $r$.
    
    Consider any simple function $s:[0,1]\to[0,1]$.
    By definition, it is of the form:
    \[
      s(x)=\sum_{i=1}^m r_i 1_{A_i}(x)
    \]
    where $m<\omega$, $r_1,\dots,r_m\in[0,1]$, $A_1,\dots,A_m$ disjoint Borel sets, and $1_{A_1},\dots,1_{A_m}$ are the indicator functions for $A_1,\dots,A_m$.
    We see then that the sentence
    \[
      \bigvee_{i=1}^m \num{r}_i\wedge\chi_{A_i}
    \]
    is equivalent to $s$ when considered as a function of $R$.
    
    We therefore have all simple functions, so by approximating $f$ with simple functions from below and disjuncting their corresponding sentences, we achieve a sentence emulating $f$.
  \end{proof}
\end{theorem}
\begin{remark}
  The theorem above essentially states that any additional connective we add to the language would be redundant.
  The above result is interesting but a bit heavy-handed.
  Caicedo and Iovino~\cite{caicedo2012omitting} demonstrated that even in the finite language, any continuous connective was redundant (in that the extended logic would be a conservative extension).
\end{remark}
\section{Model Existence}
This section attempts to adapt to our logic the ideas from chapter 3 of~\cite{keisler1971model}, a chapter which defines and proves the model existence theorem.
Specifically, we introduce an adaptation of the classical \emph{consistency property} to $[0,1]$-valued logic.
In classical logic, the consistency property is used to guide a step-by-step process whereby sentences are individually assigned a truth value.
Our approach is to assign \emph{ranges} of possible truth values, rather than specific truth values.

With this idea in mind, we introduce our first definition:
\begin{definition}\label{def:constraint}
  For any sentence $\phi$, a \emph{constraint} on $\phi$ is a sentence of the form $\rat p\to\phi$ or of the form $\phi\to\rat q$, where $p,q\in\Q\cap[0,1]$.
  A set of constraints is a constraint-theory.
  A constraint-theory $s$ is \emph{valid} if there is no sentence $\phi$ and rationals $p,q\in\Q\cap[0,1]$ such that $p>q$ and $\{\rat p\to\phi,\phi\to\rat q\}\subseteq s$.
  We say a formula $\phi$ is \emph{constrained by} $s$, if $s$ is a constraint-theory that contains a constraint on $\phi$.
\end{definition}

We also need a definition for technical reasons that will help us later in the proof of the Model Existence Theorem.
We already defined an arrow-manipulation in Definition~\ref{def:arrow-manipulation}.
Arrow-manipulations are restricted to be only one step away from the sentence they are equivalent to; however, we need to be able to talk about formulas which are equivalent but are multiple steps away.
In other words, we need some extended notion of arrow-manipulation (as well as a definition for those sentences without extended arrow-manipulations), which we achieve with the definitions below.
\begin{definition}
  A formula is \emph{basic} if its only connectives are `$\to$' and the rationals (ie.\ the formula does not contain $\unwedge$, $\unvee$, $\forall$, or $\exists$).
\end{definition}
\begin{definition}
	For any formula $\phi$, we define the set of \emph{extended arrow-manipulations} of $\phi$ inductively as follows:
  \begin{itemize}
    \item The formula $\phi$ is itself an extended arrow-manipulation of $\phi$.
    \item All arrow-manipulations of $\phi$ are extended arrow-manipulations of $\phi$.
    \item If $\phi$ is of the form $\phi_1\to\phi_2$ then:
    \begin{itemize}
      \item 
        For any $\psi_1$ that is an extended arrow-manipulation of $\phi_1$, the formula $\psi_1\to\phi_2$ and all arrow-manipulations of $\psi_1\to\phi_2$ are extended arrow-manipulations of $\phi$.
      \item 
        For any $\psi_2$ that is an extended arrow-manipulation of $\phi_2$, the formula $\phi_1\to\psi_2$ and all arrow-manipulations of $\phi_1\to\psi_2$ are extended arrow-manipulations of $\phi$.
    \end{itemize}
  \end{itemize}
\end{definition}
\begin{remark}
  Let $\phi$ be a formula.
  One can show by induction that all extended arrow-manipulations of $\phi$ are in every fragment containing $\phi$.
  Additionally, one can show that all extended arrow-manipulations of $\phi$ are semantically equivalent to $\phi$.
\end{remark}
The above remark is merely housekeeping; Lemma~\ref{lem:extended-arrow-manipulations} is the true motivation for the above definition.
Before introducing the lemma, we need to introduce a notion of complexity that is needed only to make the induction go through.
\begin{remark}\label{rem:complexity}
  We define the complexity of $\phi$ inductively the same way as in classical $\lang$, except that for any formulas $\phi$ and $\psi$, the complexity of $\phi\to\psi$ is the sum of the complexities of $\phi$ and $\psi$.
  This notion of complexity is required exactly twice: for the induction in the model existence theorem and for the statement of Lemma~\ref{lem:extended-arrow-manipulations}.
  As it is needed nowhere else, we do not give a proper name to it nor take it to be the true definition of complexity.
  When it is used, we will simply say ``the complexity defined in Remark~\ref{rem:complexity}.''
\end{remark}
\begin{lemma}\label{lem:extended-arrow-manipulations}
  For any non-basic formula $\phi$, there is an extended arrow-manipulation of $\phi$ that does not have `$\to$' as its top-level connective and is of no greater complexity than $\phi$ (when using the complexity defined in Remark~\ref{rem:complexity}).
  \begin{proof}
    The proof is by induction on the complexity of formulas, as defined in Remark~\ref{rem:complexity}.
    There is no non-basic formula with complexity 1, so there is no base case.
    
    Let $\phi$ be a non-basic formula such that all formulas of lesser complexity satisfy the induction.
    If $\phi$ is not of the form $\phi_1\to\phi_2$, then it doesn't have a top-level `$\to$' and therefore satisfies this lemma.
    
    Assume that $\phi$ does have the form $\phi_1\to\phi_2$ for some formulas $\phi_1$ and $\phi_2$.
    We know that $\phi$ is non-basic, so one of $\phi_1$ or $\phi_2$ is non-basic.
    
    Assume that $\phi_1$ is non-basic.
    By induction, there is some extended arrow-manipulation $\psi$ of $\phi_1$ with no greater complexity that $\phi_1$ and that is of one of these forms:
    \begin{align*}
      \unwedge\Phi &&
      \unvee\Phi &&
      \forall x\,\chi(x) &&
      \exists x\,\chi(x) 
    \end{align*}
    
    Assume that $\psi$ is of the form $\unwedge\Phi$ where $\Phi$ is a countable set of formulas.
    By definition of extended arrow-manipulation, we know that all arrow-manipulations of $(\unwedge\Phi)\to\phi_2$ are extended arrow-manipulations of $\phi$.
    Specifically, we see that
    \[
      \bigvee_{\chi\in\Phi}(\chi\to\phi_2)
    \]
    is an extended arrow-manipulation of $\phi$, and its complexity is bounded by the complexity of $\phi$.
    The case for $\psi$ being $\unvee\Phi$ is similar.
    
    Assume that $\psi$ is of the form $\forall x\,\chi(x)$ for some formula $\chi$.
    Without loss of generality, assume that $x$ is not free in $\phi_2$.
    We notice that
    \[
      \exists x\,(\chi(x)\to\phi_2)
    \]
    is an arrow-manipulation of $\psi\to\phi_2$ and is thus an extended arrow-manipulation of $\phi$.
    Notice that the above formulas complexity is the same as the complexity of $\phi$, so we are done.
    The case for $\psi$ being $\exists x\,\chi(x)$ is similar.
    
    The case for $\phi_2$ being non-basic instead of $\phi_1$ is similar.
    That concludes the induction.
  \end{proof}
\end{lemma}
We now define consistency property for $[0,1]$-valued logic.
As for the classical setup, we have a rule for every connective and a rule for consistency.
In addition, we add the ``Narrowing'' and ``Introduction'' rules.

\begin{definition}\label{def:cp}
	\newcommand\litem[1]{\item{(#1)}}
	We say a non-empty set $\mathscr{S}$ of countable, valid constraint-theories is a \emph{consistency property} if for all $s\in \mathscr{S}$ and $p,q,r\in\Q\cap[0,1]$, we have the following:
  \begin{enumerate}[label=(CP\arabic*)]
  \litem{Consistency Rule}\label{itm:cp-rule-consistency}
    For any \emph{basic} sentences $\theta_1$ and $\theta_2$, we have the following:
    \begin{itemize}
    \item If $s$ contains $\rat p\to\theta_1$, $\theta_2\to\rat q$, and $\rat r\to(\theta_1\to\theta_2)$, then $r\leq(p\Rightarrow q)$.
%    $\rat p\to\theta_1$ and $\theta_2\to\rat q$ are in $s$, then $s\cup\{(\theta_1\to\theta_2)\to\rat{p\Rightarrow q}\}$ is in $S$.
    \item If $s$ contains $\theta_1\to\rat p$, $\rat q\to\theta_2$, and $(\theta_1\to\theta_2)\to\rat r$, then $r\geq(p\Rightarrow q)$.
    %If $\theta_1\to\rat q$ and $\rat p\to\theta_2$ are in $s$, then $s\cup\{\rat{q\Rightarrow p}\to(\theta_1\to\theta_2)\}$ is in $S$.
    %\item If $\theta_1\to\theta_2$ is constrained by $s$, then both $s\cup\{\rat 0\to\theta_1\}$ and $s\cup\{\rat 0\to\theta_2\}$ are in $S$.
    \end{itemize}
  \litem{$\to$-Rule}\label{itm:cp-rule-pushdown}
    For any sentence $\phi$, if $\rat p\to\phi$ is in $s$, then $s\cup\{\rat p\to\psi\}$ is in $\mathscr{S}$ for every extended arrow-manipulation $\psi$ of $\phi$.
  \litem{$\wedge$-Rule}\label{itm:cp-rule-wedge}
    For any countable set of sentences $\Phi$, if $\rat p\to\unwedge\Phi$ is in $s$, then for all $\phi\in\Phi$, the set $s\cup\{\rat p\to\phi\}$ is in $\mathscr{S}$.
	\litem{$\vee$-Rule}\label{itm:cp-rule-vee}
		For any countable set of sentences $\Phi$, if $\rat p\to\unvee\Phi$ is in $s$, then for all $p_0\in(\Q\cap[0,p))\cup\{0\}$, there is some $\phi\in\Phi$ such that $s\cup\{\rat{p_0}\to\phi\}\in \mathscr{S}$.
	\litem{$\forall$-Rule}\label{itm:cp-rule-forall}
	  For any formula $\phi(x)$, if $\rat{p}\to\forall x\,\phi(x)$ is in $s$, then for any closed term $t$, the set $s\cup\{\rat p\to\phi(t)\}$ is in $\mathscr{S}$
	\litem{$\exists$-Rule}\label{itm:cp-rule-exists}
	  For any formula $\phi(x)$, if $\rat p\to\exists x\,\phi(x)$ is in $s$, then for all $p_0\in(\Q\cap[0,p))\cup\{0\}$, there is some closed term $t$ such that $s\cup\{\rat{p_0}\to\phi(t)\}\in \mathscr{S}$.
  \litem{$\Q$-Rule}\label{itm:cp-rule-q}
    The set $s\cup\{\rat q\to\rat q\}$ is in $\mathscr{S}$.
  \litem{Introduction Rule}\label{itm:cp-rule-intro}
    For any \emph{basic} sentences $\theta_1$ and $\theta_2$, if $\theta_1\to\theta_2$ is in $s$, then $s\cup\{\rat 0\to\theta_1,\rat 0\to\theta_2\}$ is in $\mathscr{S}$.
  \litem{Narrowing Rule}\label{itm:cp-rule-narrowing}
    For any sentence $\phi$, if $s$ contains a bound on $\phi$, then for all $\epsilon>0$, there is some $p_0,q_0\in\Q\cap[0,1]$ with $p_0\leq q_0< p_0+\epsilon$ such that
    \[
      s\cup\{\rat{p_0}\to\phi,\ \phi\to\rat{q_0}\}
    \]
    is in $\mathscr{S}$.
	\end{enumerate}
\end{definition}

We now introduce some original notation which will help us talk about to where the ranges are converging.
It is in the same spirit as the Pavelka provability degree that will be introduced in Definition~\ref{def:proof-degree}, and so we name and denote it similarly.
\begin{definition}
  For any constraint-theory $s$ and sentence $\phi$, the \emph{Pavelka constraint degree} of $\phi$ in $s$, denoted $|\phi|^*_s$, is the value
  \[
    |\phi|^*_s:=\sup\{p\in\Q\cap[0,1]\where\text{$p=0$ or $(\rat p\to\phi)\in s$}\}.
  \]
  Note that $|\phi|^*_s$ is always in $[0,1]$ and may be irrational.
\end{definition}
The lemma below is used as a single-step in the inductive process used in the model existence theorem proof.
\begin{lemma}\label{lem:one-step}
  Let $\mathscr{S}$ be a consistency property.
  If $\phi$ is a sentence and $s$ is a constraint-theory in $\mathscr{S}$ such that $\phi$ is constrained by $s$, then for all $\epsilon>0$, there is some $s^*\in \mathscr{S}$ such that the following hold:
  \begin{itemize}
  \item 
    There is $p,q\in\Q\cap[0,1]$ with $p\leq q<p+\epsilon$ such that $\{\rat p\to\phi,\ \phi\to\rat q\}$ is a subset of $s^*$.
  \item 
    If $\phi$ is of the form $\unvee\Phi$, then there is some $\psi\in\Phi$ and $p\in\Q\cap[0,1]$ with $p>|\phi|^*_s-\epsilon$ such that $\rat p\to\psi$ is in $s^*$.
  \item 
    If $\phi$ is of the form $\exists x\,\psi(x)$, then there is some closed term $t$ and some rational $p\in\Q\cap[0,1]$ with $p>|\phi|^*_s-\epsilon$ such that $\rat p\to\psi(t)$ is in $s^*$.
  \end{itemize}
  \begin{proof}
    Fix $\phi$ and $s$ as above.
    Let $\epsilon>0$ be given.
    
    By the Narrowing Rule~\ref{itm:cp-rule-narrowing}, we know that we can choose some $p_1,q_1\in\Q\cap[0,1]$ and $s_1\in \mathscr{S}$ such that $p_1\leq q_1 <p_1+\epsilon$ and\[
      s_1:=s\cup\{\rat{p_1}\to\phi,\ \phi\to\rat{q_1}\}.
    \]
    
    We know that the set $\{p\in\Q\cap[0,1]\where(\rat p\to\phi)\in s_1\}$ contains $p_1$ and is therefore non-empty.
    Thus, the definition of $|\phi|^*_{s_1}$ gives us some $p_2\in\Q\cap[0,1]$ with $p_2>|\phi|^*_{s_1}-\tfrac\epsilon2$ such that the constraint $\rat{p_2}\to\phi$ is in $s_1$.
    Because $s$ is a subset of $s_1$, we know $|\phi|^*_s\leq|\phi|^*_{s_1}$, which implies that $p_2>|\phi|^*_s-\frac\epsilon2$.
    
    If $p_2=0$, then fix $p:=0$.
    Otherwise, choose $p\in\Q\cap[0,1]$ such that 
    \[
      p_2-\tfrac\epsilon2<p< p_2.
    \]
    Notice $p>p_2-\tfrac\epsilon2>|\phi|^*_s-\epsilon$.
    
    We now define $s^*\supseteq s_1$ by casing on the shape of $\phi$.
    If $\phi$ is of the form $\unvee\Phi$, then invoking the $\unvee$-Rule \ref{itm:cp-rule-vee} allows us to choose some $\psi\in\Phi$ such that $s^*:=s_1\cup\{\rat p\to\psi\}\in \mathscr{S}$.
    If $\phi$ is of the form $\exists x\,\psi(x)$, then invoking the $\exists$-Rule \ref{itm:cp-rule-exists} allows us to choose some closed term $t$ such that $s^*:=s_1\cup\{\rat p\to\psi(t)\}\in \mathscr{S}$.
    If $\phi$ is neither of these forms, then take $s^*:=s_1$.
    In any case, we've defined $s^*$ satisfying the desired three properties.
  \end{proof}
\end{lemma}

\begin{theorem}[Model Existence Theorem]
  If $\mathscr{S}$ is a consistency property, then for every $s\in \mathscr{S}$, there is a model realizing the theory $\{\phi\where |\phi|^*_s=1\}$.
  \begin{proof}
    Let $s\in \mathscr{S}$ be given and take $\frag$ to be a countable fragment (see Definition~\ref{def:fragment}) containing all sentences constrained by $s$.
    Fix a sequence $(\phi_n)$ of $\mathcal L$-sentences such that every sentence in $\mathcal L$ appears infinitely often.
    
    We can define a chain of constraint-theories $s=s_0\subseteq s_1\subseteq\cdots\in \mathscr{S}$ as follows:
    \begin{itemize}
    \item If $\phi_n$ is $\rat q$ for some $q\in\Q\cap[0,1]$, then $s_{n+1}=s_n\cup\{\rat q\to\rat q\}$.
    \item If $\phi_n$ is not constrained by $s_n$ but there is some $s^*\in \mathscr{S}$ such that $s^*\supseteq s_n$ and $\phi_n$ is constrained by $s^*$, then $s_{n+1}$ is some such $s^*$.
    \item If $\phi_n$ is not constrained by $s_n$ and there is no superset of $s_n$ in $\mathscr{S}$ that constrains $\phi_n$, then $s_{n+1}=s_n$.
    \item If $\phi_n$ is constrained by $s_n$, then using Lemma~\ref{lem:one-step}, we choose $s_{n+1}$ such that all of the following hold:
	    \begin{itemize}
		  \item 
		    There is $p,q\in\Q\cap[0,1]$ with $p\leq q<p+\tfrac1{n+1}$ such that both $\rat p\to\phi_n$ and $\phi_n\to\rat q$ are in $s_{n+1}$.
		  \item 
		    If $\phi_n$ is of the form $\unvee\Phi$, then there is some $\psi\in\Phi$ and $p\in\Q\cap[0,1]$ with $p>|\phi|^*_s-\tfrac1{n+1}$ such that $\rat p\to\psi$ is in $s_{n+1}$.
		  \item 
		    If $\phi_n$ is of the form $\exists x\,\psi(x)$, then there is some closed term $t$ and some rational $p\in\Q\cap[0,1]$ with $p>|\phi|^*_s-\tfrac{1}{n+1}$ such that $\rat p\to\psi(t)$ is in $s_{n+1}$.
		  \end{itemize}
    \end{itemize}	  
	  Define $s_\omega:=\bigcup\limits_{n<\omega}s_n$.
	  Clearly $s_\omega$ is a countable and valid constraint-theory.
	  
	  We will now use $s_\omega$ to define a structure $\mathcal M$ realizing $\{\phi\where|\phi|^*_s=1\}$.
    
    Define the universe of $\mathcal M$ to be the set of closed terms in the language.
    For each $n$-ary function symbol $f$ in the language and $n$-tuple of terms $\V t$, define $f^\mathcal M(\V t^\mathcal M)$ as the term $f(\V t)$.
    For each $n$-ary relation symbol $R$ in the language and $n$-tuple of terms $\V t$, define $R^\mathcal M(\V t^\mathcal M)$ as $|R(\V t)|^*_{s_\omega}$.
    
    That completes the definition of $\mathcal M$.
    
    The remainder of this proof is dedicated to showing that for each sentence $\phi$ in the fragment that is constrained by $s_\omega$, we have $|\phi|^*_{s_\omega}\leq\phi^\mathcal M$.
    If this were true, then we would know
    \[
      \mathcal M\models\{\phi\where |\phi|^*_s=1\}
    \]
    because $|\phi|^*_s\leq|\phi|^*_{s_\omega}$ for any sentence $\phi$ constrained by $s$.
    
    We split the proof into three claims.
	  \begin{claim}{1}
	  If $\phi$ is an $\mathcal L$-sentence constrained by $s_\omega$, then for all $\epsilon>0$, there is some $q\in\Q\cap[0,1]$ with $|\phi|^*_{s_\omega}\leq q<|\phi|^*_{s_\omega}+\epsilon$ such that the constraint $\phi\to\rat q$ is in $s_\omega$.
	  \begin{claimproof}
	    Let $\phi$ be an $\mathcal L$-sentence constrained by $s_\omega$, which means there is some $N<\omega$ such that $\phi$ is constrained by $s_N$.
	    Because $\phi$ appears infinitely often in the sequence $(\phi_n)$, there is some $n>N$ such that $\phi_n$ is $\phi$ and $n$ is large enough so that $\tfrac1{n+1}<\epsilon$.
	    Because $s_n\supseteq s_N$, we know that $\phi$ is constrained by $s_n$, so there is $p,q\in\Q\cap[0,1]$ with $p\leq q<p+\tfrac1{n+1}<p+\epsilon$ such that $\{\rat p\to\phi,\ \phi\to\rat q\}\subseteq s_{n+1}$.
	    This means $q<|\phi|^*_{s_\omega}+\epsilon$ because $p$ and $q$ are within $\epsilon$ and $|\phi|^*_{s_\omega}$ is between $p$ and $q$.
	    Therefore, the constraint $\phi\to\rat q$ satisfies the claim.
	  \end{claimproof}
	  \end{claim}
    \begin{claim}{2}
      For any basic $\mathcal L$-sentence $\phi$ constrained by $s_\omega$, we have $|\phi|^*_{s_\omega}=\phi^\mathcal M$.
      \begin{claimproof}
        The proof goes by induction on the complexity of basic sentences.
        
        For atomic sentences of the form $R(\V t)$, where $R$ is a relation and $\V t$ is a tuple of terms, we know that $R^\mathcal M(\V t^\mathcal M)$ is defined to be exactly $|R(\V t)|^*_{s_\omega}$.
        For atomic sentences of the form $\rat q$ where $q$ is a rational, consider that the constraint $\rat q\to\rat q$ is in $s_\omega$, so $|\rat q|^*_{s_\omega}=q$.
        These are the only two types of atomic sentences.
        
        We move onto the inductive case.
        Assume that we have a basic sentence $\theta_1\to\theta_2$ constrained by $s_\omega$.
        This means there is some $m<\omega$ such that $\theta_1\to\theta_2$ is constrained by $s_m$.
        The Introduction Rule~\ref{itm:cp-rule-intro} guarantees that for any superset of $s_m$ in $\mathscr{S}$, the set can be extended to include constraints on $\theta_1$ and $\theta_2$.
        Thus, we know $\theta_1$ and $\theta_2$ are constrained by $s_\omega$, so we apply the inductive hypothesis to see that $|\theta_1|^*_{s_\omega}=\theta_1^\mathcal M$ and $|\theta_2|^*_{s_\omega}=\theta_2^\mathcal M$.
        
        Because $[\theta_1\to\theta_2]^\mathcal M$ is defined to be $\min\{1,1-\theta_1^\mathcal M+\theta_2^\mathcal M\}$, we need only show that $|\theta_1\to\theta_2|^*_{s_\omega}=\min\{1,1-|\theta_1|^*_{s_\omega}+|\theta_2|^*_{s_\omega}\}$.
        
        Assume for the sake of contradiction that $|\theta_1\to\theta_2|^*_{s_\omega}>\min\{1,1-|\theta_1|^*_{s_\omega}+|\theta_2|^*_{s_\omega}\}$.
        Utilizing Claim~1 and some basic arithmetic manipulation, we may take rationals $p,q,r\in\Q\cap[0,1]$ such that $r>(p\Rightarrow q)$ and
        \[
          \{\theta_1\to\rat p,\ \rat q\to\theta_2,\ \rat r\to(\theta_1\to\theta_2)\}\subseteq s_{\omega},
        \]
        but the set on the left is finite, so we may take $n<\omega$ such that
        \[
          \{\theta_1\to\rat p,\ \rat q\to\theta_2,\ \rat r\to(\theta_1\to\theta_2)\}\subseteq s_n,
        \]
        which does not satisfy the Consistency Rule~\ref{itm:cp-rule-consistency}.
        We have arrived at our contradiction, so $|\theta_1\to\theta_2|^*_{s_\omega}\leq\min\{1,1-|\theta_1|^*_{s_\omega}+|\theta_2|^*_{s_\omega}\}$
        
        The proof of $|\theta_1\to\theta_2|^*_{s_\omega}\geq\min\{1,1-|\theta_1|^*_{s_\omega}+|\theta_2|^*_{s_\omega}\}$ is similar, and left out for brevity.
        From the two inequalities, we see
        \[
          |\theta_1\to\theta_2|^*_{s_\omega}=\min\{1,1-|\theta_1|^*_{s_\omega}+|\theta_2|^*_{s_\omega}\}=:(|\theta_1|_{s_\omega}^*\Rightarrow|\theta_2|_{s_\omega}^*),
        \] and thus conclude the induction.
      \end{claimproof}
    \end{claim}
    \begin{claim}{3}
      For any $\mathcal L$-sentence $\phi$ constrained by $s_\omega$, we have $|\phi|^*_{s_\omega}\leq\phi^\mathcal M$.
      \begin{claimproof}
        We prove this claim by induction on the complexity of sentences, using the complexity defined in Remark~\ref{rem:complexity}.
        Claim~2 showed it true for basic sentences, so we need to examine only the non-basic cases.
        
        Let $\phi$ be a non-basic sentence such that all sentences of lesser complexity satisfy the claim.
        We case on the shape of $\phi$.
        \begin{itemize}
        \item
          Assume that $\phi$ is of the form $\psi_1\to\psi_2$, so by Lemma~\ref{lem:extended-arrow-manipulations}, we know there is some extended arrow-manipulation $\psi$ of $\phi$ such that $\psi$ is of no greater complexity than $\phi$ and is not an implication.
          Utilizing the $\to$-Rule~\ref{itm:cp-rule-pushdown}, we can show that $|\phi|^*_{s_\omega}\leq|\psi|^*_{s_\omega}$.
          Showing $|\psi|^*_{s_\omega}\leq \psi^\mathcal M$ is handled by the other cases.
          Combining that with the fact that extended arrow-manipulations are semantically equivalent (in particular, $\psi^\mathcal M=\phi^\mathcal M$), we get that $|\phi|^*_{s_\omega}\leq\phi^\mathcal M$.
        \item
          Assume $\phi$ is of the form $\unwedge\Phi$. 
          By the $\unwedge$-Rule~\ref{itm:cp-rule-wedge}, we can show that $|\phi|^*_{s_\omega}\leq|\psi|^*_{s_\omega}$ for all $\psi\in\Phi$.
          Applying the induction hypothesis, we see:
          \[
            |\phi|^*_{s_\omega}\leq\inf_{\psi\in\Phi}|\psi|^*_{s_\omega}\leq\inf_{\psi\in\Phi}\psi^\mathcal M=:\phi^\mathcal M
          \]
        \item
          Assume $\phi$ is of the form $\unvee\Phi$.
          By the definition of the chain $s_0\subseteq s_1\subseteq\cdots$, we know that there are arbitrarily large $n<\omega$ with some $\psi\in\Phi$ (dependent on $n$) with $|\psi|^*_{s_\omega}>|\phi|^*_{s_\omega}-\tfrac1{n+1}$.
          Hence:
          \[
            |\phi|^*_{s_\omega}\leq\sup_{\psi\in\Phi}|\psi|^*_{s_\omega}\leq\sup_{\psi\in\Phi}\psi^\mathcal M=\phi^\mathcal M
          \]
        \item
          Assume $\phi$ is of the form $\forall x\,\psi(x)$.
          By the $\forall$-Rule~\ref{itm:cp-rule-forall}, we can show that $|\phi|^*_{s_\omega}\leq|\psi(t)|^*_{s_\omega}$ for every closed term $t$.
          So:
          \[
            |\phi|^*_{s_\omega}\leq\inf_t|\psi(t)|^*_{s_\omega}\leq\inf_t[\psi(t)]^\mathcal M\leq\phi^\mathcal M
          \]
        \item
          Assume $\phi$ is of the form $\exists x\,\psi(x)$.
          By the definition of $s_\omega$, we know that there are arbitrary large $n<\omega$ such that there is closed some term $t$ with $|\psi(t)|^*_{s_\omega}>|\phi|^*_{s_\omega}-\tfrac1{n+1}$.
          Hence:
          \[
            |\phi|^*_{s_\omega}\leq\sup_t|\psi(t)|^*_{s_\omega}\leq\sup_t[\psi(t)]^\mathcal M=\phi^\mathcal M
          \]
        \end{itemize}
        That covers all the possibilities of $\phi$, so we conclude the induction.
      \end{claimproof}
    \end{claim}
    
    Claim~3 is enough, as argued before introducing any of the claims, to achieve the desired result.
  \end{proof}
\end{theorem}

\begin{corollary}[Extended Model Existence]\label{thm:model-existence-ex}
  Let $T$ be a countable theory and $\mathscr{S}$ a consistency property.
  If for all $s\in \mathscr{S}$, $\phi\in T$, and $p\in\Q\cap[0,1)$, the set $s\cup\{\rat p\to \phi\}$ is in $\mathscr{S}$, then there is a model of $T$.
  \begin{proof}
    Take $s^*=\{\rat p\to\phi\where\phi\in T,\ p\in\Q\cap[0,1)\}$.
    It is straightforward to show that the set $\{s\cup s^*\where s\in \mathscr{S}\}$ is a consistency property.
    Fixing any $s\in \mathscr{S}$, we can invoke Model Existence to show that $\{\phi\where |\phi|_{s\cup s^*}=1\}$ has a model, which gives a model of $T$.
  \end{proof}
\end{corollary}

\section{Completeness}
As the section title suggests, this section is devoted to proving completeness.
That completeness holds for finite theories in {\luk}-Pavelka logic is well-known, as well as Pavelka-style completeness for all theories.
This section serves two purposes: to show completeness for \emph{infinitary} $[0,1]$-valued logic and to demonstrate a use-case for the model existence theorem.
To accomplish this, we use the Henkin construction in chapter 4 of~\cite{keisler1971model}.

\begin{definition}[Inference Rules]
  For any formulas $\phi$ and $\psi$ and any countable set of formulas $\Phi$, we have the inference rules:
  \begin{align*}
    \infer{\psi}{\phi & \phi\to\psi} &&
    \infer{\unwedge\Phi}{\phi\text{ (for all $\phi\in\Phi$)}} &&
    \infer{\forall x\,\phi}{\phi\text{ ($x$ not free in assumptions in proof of $\phi$)}}
  \end{align*}
  When these inference rules are used in proof-theoretic proofs, we refer to them respectively as Arrow-Elimination~($\arr\elim$), And-Introduction~($\unwedge\intro$), and Generalization~($\generalization$); though we often write modus ponens instead of Arrow-Elimination.
\end{definition}
\begin{definition}
  The following comprise the axiom-schema of our logic (see Notation~\ref{not:shorthand} and Notation~\ref{not:strict} for the definition of $\neg$, $\leftrightarrow$, and $\strict{\cdot}$):
	\begin{enumerate}[label=(A\arabic*)]
		\item\label{itm:axiom-weakening}
			$\phi\to(\psi\to\phi)$
		\item\label{itm:axiom-trans}
			$(\phi\to\psi)\to((\psi\to\chi)\to(\phi\to\chi))$
		\item
		  $(\neg\phi\to\neg\psi)\to(\psi\to\phi)$
		\item\label{itm:axiom-vee}
			$((\phi\to\psi)\to\psi)\leftrightarrow(\phi\vee\psi)$
		\item\label{itm:axiom-rationals}
			$(\rat p\rightarrow\rat q)\leftrightarrow \rat r$ where $p,q,r\in\Q\cap[0,1]$ such that $r=(p\Rightarrow q)$
		\item\label{itm:axiom-forall-arrow}
			$(\forall x\,\phi(x))\rightarrow\phi(t)$
		\item\label{itm:axiom-and}
			$(\unwedge\Phi)\rightarrow\phi$ for all $\phi\in\Phi$
		\item\label{itm:axiom-pushdown}
			$\phi\leftrightarrow\psi$ where $\psi$ is an arrow-manipulation of $\phi$ (recall Definition~\ref{def:arrow-manipulation})
	  \item\label{itm:axiom-loe}
      $\strict\phi\vee\neg\strict\phi$
	  \item\label{itm:axiom-approach}
      $\bigvee_{p\in\Q\cap[0,1)}\rat p$	  
		\end{enumerate}
\end{definition}
\begin{theorem}[Soundness]
  The inference rules are sound and the axioms above are tautologies (evaluate to 1 in all structures).
  \begin{proof}
    The proof is by invoking semantic definitions and is left to the reader.
  \end{proof}
\end{theorem}
\begin{remark}\label{rem:auto-commutive}
  In most axiomatizations of {\L}ukasiewicz logic, one would include the commutativity of $\vee$, namely:
  \[
    ((\phi\to\psi)\to\psi)\to((\psi\to\phi)\to\phi)
  \]
  However, we have defined $\phi\vee\psi$ as shorthand for $\unvee\{\phi,\psi\}$, and so it is automatically commutative, allowing the sentence above to be derived trivially from~\ref{itm:axiom-vee}.
\end{remark}
\begin{remark}
  Axiom~\ref{itm:axiom-approach} is actually redundant in the general proof system (as will be seen in the proof of Theorem~\ref{thm:completeness-in-fragments}), though we have not shown it to be so in all fragments, so we leave it as an axiom.
\end{remark}
\begin{notation}
	As with all proofs of completeness, there are occasional proof-theoretic proofs scattered throughout this section. 
	To help the reader parse these basic proofs, we opt for the notation of proof trees.
	
	To keep the proofs clean, for axioms \ref{itm:axiom-vee}, \ref{itm:axiom-rationals}, and \ref{itm:axiom-pushdown} (ie.\ the axioms which are two-way implications), we will invoke each direction of the arrow individually instead of writing down the axiom and then showing it implies the direction necessary for the rest of the proof.
	In other words, we will write
	\[
	  \infer[\aref{rationals}]{(\rat p\to\rat q)\to\rat r}{}
	\]
	instead of
	\[
	  \infer[\arr\elim]{(\rat p\to\rat q)\to\rat r}{
	    \infer[\aref{rationals}]{(\rat p\to\rat q)\leftrightarrow\rat r}{} &
	    \infer[\aref{and}]{((\rat p\to\rat q)\leftrightarrow\rat r)\to((\rat p\to\rat q)\to\rat r)}{}
	  }
	\]
\end{notation}
\begin{lemma}\label{lem:near-axioms}
  The following are derivable using only the axioms and modus ponens:
  \begin{enumerate}[label=(PL\arabic*), font=\normalfont]
  \item\label{itm:axiom-vee-arrow-def}$\phi\to((\phi\to\psi)\to\psi)$
  	\item\label{itm:axiom-swap}$(\phi\to(\psi\to\chi))\to(\psi\to(\phi\to\chi))$
  \item\label{itm:axiom-id-arrow} $\phi\to\phi$
  \item\label{itm:axiom-0-arrow} $\rat0\to\phi$
  \item\label{itm:axiom-classic}$\neg\neg\phi\to\phi$
  \item\label{itm:axiom-vee-arrow} $\phi\to(\phi\vee\psi)$
  \item\label{itm:axiom-rev-classic} $\phi\to\neg\neg\phi$
  \item\label{itm:axiom-rev-trans} $(\psi\to\chi)\to((\phi\to\psi)\to(\phi\to\chi))$
  \item\label{itm:axiom-order}$(\phi\to\psi)\vee(\psi\to\phi)$
  \item\label{itm:axiom-1} $\rat 1$
  \item\label{itm:axiom-arrow-explode} $\neg\phi\to(\phi\to\psi)$
  \item\label{itm:axiom-strict-arrow} $\strict\phi\to\phi$
  \end{enumerate}
  \begin{proof}
    Observe {\L}ukasiewicz's original axioms\footnote{{\L}ukasiewicz also took $(\phi\to\psi)\vee(\psi\to\phi)$ as an axiom, but~\cite{chang1958proof} showed it redundant.}:
    \begin{itemize}
    \item $\phi\to(\psi\to\phi)$
    \item $(\phi\to\psi)\to((\psi\to\chi)\to(\phi\to\chi))$
    \item $(\neg\phi\to\neg\psi)\to(\psi\to\phi)$
    \item $((\phi\to\psi)\to\psi)\to((\psi\to\phi)\to\phi)$
    \end{itemize}
    We took the first three as axioms and showed the fourth in Remark~\ref{rem:auto-commutive}.
    We refer to Lemma~3.1.6 and Theorem~3.1.15 of~\cite{hajek1998metamathematics} for proofs of \ref{itm:axiom-vee-arrow-def} through \ref{itm:axiom-order} from the four statements above.
    We prove the other schemas here.
    
    The proof of $\rat 1$ comes from \aref{id-arrow} and $(\rat 1\to\rat 1)\to\rat 1$, which is an instance of \aref{rationals}.
    The proof of $\neg\phi\to(\phi\to\psi)$ is:
    \[
      \infer[\arr\elim]{\neg\phi\to(\phi\to\psi)}{
        \infer=[\aref{0-arrow}]{\rat 0\to\psi}{} &
        \infer=[\aref{rev-trans}]{(\rat 0\to\psi)\to(\neg\phi\to(\phi\to\psi))}{}
      }
    \]
    Recall Notation~\ref{not:strict} states that $\strict\phi$ is shorthand for a conjunction containing $\neg(\phi\to\rat 0)$, which is just $\neg\neg\phi$. 
    Therefore, the proof of $\strict\phi\to\phi$ is:
    \[
      \infer[\arr\elim]{\strict\phi\to\phi}{
	      \infer[\aref{and}]{\strict\phi\to\neg\neg\phi}{} &
	      \infer[\arr\elim]{(\strict\phi\to\neg\neg\phi)\to(\strict\phi\to\phi)}{
	        \infer=[\aref{classic}]{\neg\neg\phi\to\phi\vphantom{\strict\phi}}{} &
	        \infer[\aref{trans}]{(\neg\neg\phi\to\phi)\to(\strict\phi\to\neg\neg\phi)\to(\strict\phi\to\phi)}{}
	      }
	    }
    \]
    That concludes the proofs for all of the schema.
  \end{proof}
\end{lemma}

In addition to the three inference rules of our logic, we found the inference rules given by the lemma below were used frequently and that giving a name to them both decreased the size of most proofs and improved their readability.
\begin{lemma}\label{lem:inference-rules}
  The following inference rules are derivable:
  {\normalfont
  \begin{itemize}
  \item Transitivity ($\trans$):
    \[
      \infer{\phi\to\chi}{\phi\to\psi & \psi\to\chi}
    \]
  \item
    Arrow-Manipulation ($\pushdown$):
    \[
      \infer{\psi\text{ (where $\phi$ is an arrow-manipulation of $\psi$)}}{\phi}
    \]
  \item
    Or-Introduction ($\unvee\intro$):
    \[
      \infer{\unvee\Phi}{\phi\text{ (where $\phi\in\Phi$)}}
    \]
  \end{itemize}
  }
  \begin{proof}
  The $\pushdown$ rule is a simple applications of modus ponens with the axiom \ref{itm:axiom-pushdown}.
  We prove the other two:
    \begin{itemize}
    \item Transitivity:
      \[
	      \infer[\arr\elim]{\phi\to\chi}{
	        \psi\to\chi &
	        \infer[\arr\elim]{(\psi\to\chi)\to(\phi\to\chi)}{
	          \phi\to\psi &
	          \infer[\aref{trans}]{(\phi\to\psi)\to((\psi\to\chi)\to(\phi\to\chi))}{}
	        }
	      }
      \]
	  \item Or-Introduction (assume $\phi\in\Phi$):
	    \[
	      \infer[\trans]{\unvee\Phi}{
	        \infer[\pushdown]{\neg\neg\unvee\Phi}{
	          \infer[\trans]{\neg\bigwedge_{\psi\in\Phi}\neg\psi}{
	            \infer[\aref{and}]{(\bigwedge_{\psi\in\Phi}\neg\psi)\to\neg\phi}{} &
	            \infer[\arr\elim]{\neg\neg\phi}{
	              \phi &
	              \infer=[\aref{rev-classic}]{\phi\to\neg\neg\phi}{}
	            }
	          }
	        } &
	        \infer=[\aref{classic}]{\neg\neg\unvee\Phi\to\unvee\Phi}{}
	      }
	      \qedhere
	    \]
    \end{itemize}
  \end{proof}
\end{lemma}
\begin{lemma}\label{lem:vee-with-strict}
  The following holds for any formulas $\phi$ and $\psi$:
  \begin{itemize}
  \item $\phi\vee\psi\proves(\phi\narrow n\rat 0)\to\psi$ for any $n<\omega$
  \item $\phi\vee\psi\proves\neg\strict\phi\to\psi$
  \end{itemize}
  \begin{proof}    
    Fix $\phi,\psi$ formulas.
    
    We prove the first point by induction on $n$.
    The $n=0$ case is just \ref{itm:axiom-0-arrow}.
    
    Now assume that $\phi\vee\psi\proves(\phi\narrow n\rat 0)\to\psi$ holds for some $n<\omega$.
    Observe:
    \[
      \infer[\arr\elim]{(\phi\narrow{n+1}\rat 0)\to(\phi\to\psi)}{
        \infer*{(\phi\narrow n\rat 0)\to\psi}{\phi\vee\psi} &
        \infer=[\aref{rev-trans}]{((\phi\narrow n\rat 0)\to\psi)\to((\phi\narrow{n+1}\rat 0)\to(\phi\to\psi))}{}
      }
    \]
    From there we have the proof:
    \[
      \infer[\trans]{(\phi\narrow{n+1}\rat 0)\to\psi}{
        \infer*{(\phi\narrow{n+1}\rat 0)\to(\phi\to\psi)}{\phi\vee\psi} &
        \infer[\arr\elim]{(\phi\to\psi)\to\psi}{
          \phi\vee\psi &
          \infer[\aref{vee}]{(\phi\vee\psi)\to((\phi\to\psi)\to\psi)}{}
        }
      }
    \]
    This completes the induction, and so the proof of the first point.
    
    We proceed to the second point with the following proof:
    \[
      \infer[\pushdown]{(\bigvee_{n<\omega}(\phi\narrow n\rat 0))\to\psi}{
        \infer[\unwedge\intro]{\bigwedge_{n<\omega}((\phi\narrow n\rat 0)\to\psi)}{
		      \infer*{(\phi\narrow n\rat 0)\to\psi\text{ (for all $n<\omega$)}}{\phi\vee\psi}
	      }
      }
    \]
    Recall that $\neg\strict\phi$ is shorthand for the formula $\neg\bigwedge_{n<\omega}\neg(\phi\narrow n\rat0)$.
    By using \ref{itm:axiom-pushdown} and \ref{itm:axiom-classic}, one can show:
    \[
      \Big(\neg\bigwedge_{n<\omega}\neg(\phi\narrow n\rat0)\Big)\leftrightarrow\Big(\neg\neg\bigvee_{n<\omega}(\phi\narrow n\rat 0)\Big)\leftrightarrow\bigvee_{n<\omega}(\phi\narrow n\rat 0)
    \]
    And so we derive $\neg\strict\phi\to\psi$, as desired.
  \end{proof}
\end{lemma}
The above result is used often in conjunction with the Deduction Theorem, but we introduce it now because it shortens the proof of the following lemma.
\begin{lemma}\label{lem:proves-strict}
  If $\phi$ is a formula, then $\phi\proves\strict{\phi}$.
  \begin{proof} 
    Fix a formula $\phi$.
    Observe the proof:
    \[
      \infer[\arr\elim]{\strict\phi}{
        \infer[\aref{loe}]{\strict\phi\vee\neg\strict\phi}{} &
        \infer[\pushdown]{(\strict\phi\vee\neg\strict\phi)\to\strict\phi}{
          \infer[\unwedge\intro]{(\strict\phi\to\strict\phi)\wedge(\neg\strict\phi\to\strict\phi)}{
            \infer=[\aref{id-arrow}]{\strict\phi\to\strict\phi}{} &
            \infer=[\text{Lemma~\ref{lem:vee-with-strict}}]{\neg\strict\phi\to\strict\phi}{
              \infer[\unvee\intro]{\phi\vee\strict\phi}{\phi}
            }
          }
        }
      }\qedhere
    \]
  \end{proof}
\end{lemma}

Recall from the first section that we do \emph{not} have contraction in $[0,1]$-valued logic.
However, we do have some sort of ``modus ponens under context'' result given by Lemma~\ref{lem:no-contraction}, but to prove that lemma we need the one below:

\begin{lemma}\label{lem:proof-technical}
  For any $n<\omega$ and formulas $\phi$, $\psi$, and $\chi$, we have the following:
  \begin{itemize}
  \item $\psi\to\chi\proves (\phi\narrow{n}\psi)\to(\phi\narrow{n}\chi)$.
  \item $\phi\to\psi\proves (\psi\narrow{n}\chi)\to(\phi\narrow{n}\chi)$.
  \end{itemize}
  \begin{proof}
    Fix $\phi$, $\psi$, and $\chi$.
    We prove both claims simultaneously by induction on $n$.
    The $n=0$ case is trivial for both points.
    Assume both claims hold for some $n\geq 0$.
    Then the first claim is shown by:
    \[
      \infer[\arr\elim]{(\phi\narrow{n+1}\psi)\to(\phi\narrow{n+1}\chi)}{
        \infer*{(\phi\narrow{n}\psi)\to(\phi\narrow{n}\chi)}{\psi\to\chi} &
        \infer=[\aref{rev-trans}]{((\phi\narrow{n}\psi)\to(\phi\narrow{n}\chi))\to((\phi\narrow{n+1}\psi)\to(\phi\narrow{n+1}\chi))}{}
      }
    \]
    The second claim requires some large formulas, so we split it into separate proofs.
    The first is:
    \[
      \infer[\arr\elim]{(\psi\narrow{n+1}\chi)\to(\phi\to(\psi\narrow n\chi))}{
        \phi\to\psi &
        \infer[\aref{trans}]{(\phi\to\psi)\to(\psi\narrow{n+1}\chi)\to(\phi\to(\psi\narrow n\chi))}{}
      }
    \]
    The second:
    \[
      \infer[\arr\elim]{(\phi\to(\psi\narrow{n}\chi))\to(\phi\narrow{n+1}\chi)}{
        \infer*{(\psi\narrow n\chi)\to(\phi\narrow n\chi)}{\phi\to\psi} &
        \infer=[\aref{rev-trans}]{((\psi\narrow n\chi)\to(\phi\narrow n\chi))\to(\phi\to(\psi\narrow n\chi))\to(\phi\narrow{n+1}\chi)}{}
      }
    \]
    The above two therefore give us:
    \[
      \infer[\trans]{(\psi\narrow{n+1}\chi)\to(\phi\narrow{n+1}\chi)}{
        \infer*{(\psi\narrow{n+1}\chi)\to(\phi\to(\psi\narrow n\chi))}{\phi\to\psi} &
        \infer*{(\phi\to(\psi\narrow{n}\chi))\to(\phi\narrow{n+1}\chi)}{\phi\to\psi}
      }
    \]
    That concludes the induction.
  \end{proof}
\end{lemma}
\begin{lemma}\label{lem:no-contraction}
  For any $m,n<\omega$ and formulas $\phi$, $\psi$, and $\chi$, we have
  \[
    \{\phi\narrow{n}\psi,\ \phi\narrow{m}(\psi\to\chi)\}\proves\phi\narrow{n+m}\chi.
  \]
  \begin{proof}  
  Fix $m$, $n$, $\phi$, $\psi$, and $\chi$.
  We start with the proof:
  \[
    \infer=[\text{Lemma~\ref{lem:proof-technical}}]{(\phi\narrow{n}\psi)\to(\phi\narrow{n}((\psi\to\chi)\to\chi))}{
      \infer=[\aref{vee-arrow-def}]{\psi\to((\psi\to\chi)\to\chi)}{}
    }
  \]
  From there, we repeatedly use the permutation axiom\footnote{While we do not take the permutation axiom as an axiom in this paper, it is an axiom of the ``basic logic'' in~\cite{hajek1998metamathematics}.}~\ref{itm:axiom-swap} to show
  \[
    \proves (\phi\narrow{n}\psi)\to(\psi\to\chi)\to(\phi\narrow{n}\chi)
  \]
  holds.
  We end this lemma with the proof:
  \[
    \infer[\trans]{\phi\narrow{n+m}\chi}{
      \phi\narrow{m}(\psi\to\chi) &
      \infer=[\text{Lemma~\ref{lem:proof-technical}}]{(\phi\narrow{m}(\psi\to\chi))\to(\phi\narrow{n+m}\chi)}{
	      \infer[\arr\elim]{(\psi\to\chi)\to(\phi\narrow{n}\chi)}{
	        \phi\narrow{n}\psi &
	        \infer*{(\phi\narrow{n}\psi)\to(\psi\to\chi)\to(\phi\narrow{n}\chi)}{}
	      }
	    }
    }\qedhere
  \]
  \end{proof}
\end{lemma}
The above lemma allows us to formulate our deduction theorem.
\begin{theorem}[Deduction]\label{thm:deduction}
  If $T$ is a set of formulas and $\phi$ and $\psi$ are formulas, then $T\cup\{\phi\}\proves\psi$ if and only if $T\proves\strict\phi\to\psi$.
  \begin{proof}
    Fix $T$ and $\phi$.
  
    We start with the right-to-left direction.
    Let $\psi$ be a formula such that $T$ proves $\strict\phi\to\psi$.
    By Lemma~\ref{lem:proves-strict}, we know $T\cup\{\phi\}\proves\strict\phi$, so one application of modus ponens gives $T\cup\{\phi\}\proves\psi$.
    That completes this direction.
    
    The other direction follows by induction on the length of proofs.
    
    For the base case, let $\psi$ be a formula that $T\cup\{\phi\}$ proves without using any inference rules.
    This means $\psi$ is either an axiom or in $T\cup\{\phi\}$.
    If $\psi$ is an axiom or in $T$, then using axiom~\ref{itm:axiom-weakening} and modus ponens, we can show $T\proves\strict\phi\to\psi$.
    If $\psi$ is $\phi$, then we reference \ref{itm:axiom-strict-arrow} to see $T\proves\strict\phi\to\phi$.
    That concludes the base case.
    
    Assume that $\psi$ is a sentence proven by $T\cup\{\phi\}$ such that all formulas before $\psi$ in the proof satisfy the hypothesis.
    We have already shown the case true when $\psi$ is not derived by inference rules, so assume that $\psi$ is inferred.

    If $\psi$ was inferred by $\unwedge\intro$, then we know that it is of the form $\unwedge\Phi$, where $\Phi$ is a countable set of formulas.
    By the induction hypothesis, we have $T\proves\strict\phi\to\chi$ for all $\chi\in\Phi$.
    Observe:
    \[
      \infer[\pushdown]{\strict\phi\to\unwedge\Phi}{
        \infer[\unwedge\intro]{\bigwedge_{\chi\in\Phi}(\strict\phi\to\chi)}{
          \strict\phi\to\chi\text{ (for all $\chi\in\Phi$)}
        }
      }
    \]
    So, $T\proves\strict\phi\to\psi$.
    
    The case for $\generalization$ is similar to the $\unwedge\intro$ case.
    
    Assume $\psi$ was inferred by modus ponens, then there must be some formula $\chi$ such that $\chi$ and $\chi\to\psi$ appear in the proof before $\psi$.
    Therefore, $T\proves\strict\phi\to\chi$ and $T\proves\strict\phi\to(\chi\to\psi)$.
    Lemma~\ref{lem:no-contraction} shows that $T\proves\strict\phi\to(\strict\phi\to\psi)$.
    So we finish this case with the proof:
    \[
      \infer[\arr\elim]{\strict\phi\to\psi}{
        \infer[\aref{loe}]{\strict\phi\vee\neg\strict\phi}{} &
        \infer[\pushdown]{(\strict\phi\vee\neg\strict\phi)\to(\strict\phi\to\psi)}{
          \infer[\unwedge\intro]{(\strict\phi\to(\strict\phi\to\psi))\wedge(\neg\strict\phi\to(\strict\phi\to\psi))}{
            \infer*{\strict\phi\to(\strict\phi\to\psi)}{T} &
            \infer=[\aref{arrow-explode}]{\neg\strict\phi\to(\strict\phi\to\psi)}{}
          }
        }
      }
    \]
    That concludes the induction and thus the theorem.
  \end{proof}
\end{theorem}
We now move into investigating how rationals interact with the proof system.
\begin{lemma}\label{lem:rationals-basic}
	The following hold for all $q\in\Q\cap[0,1]$:
  \begin{itemize}
  \item For all $p\in\Q\cap[0,1]$ and formula $\phi$, we have
	  \[
	    \proves(\rat p\to\phi)\to(\rat{q\Rightarrow p}\to(\rat q\to\phi))
	  \]
  \item For all $n<\omega$, we have
	  \[
	    \proves\rat{\min\{n(1-q),1\}}\leftrightarrow(q\narrow n\rat 0)
	  \]
  \end{itemize}
  \begin{proof}
    Fix a formula $\phi$ and rationals $p,q\in\Q\cap[0,1]$.
  
    The first point comes from using \ref{itm:axiom-swap} with the result of this proof:
    \[
      \infer[\trans]{\rat{q\Rightarrow p}\to((\rat p\to\phi)\to(\rat q\to\phi))}{
        \infer[\aref{rationals}]{\rat{q\Rightarrow p}\to(\rat q\to\rat p)}{} &
        \infer[\aref{trans}]{(\rat q\to\rat p)\to((\rat p\to\phi)\to(\rat q\to\phi))}{}
      }
    \]
    
    The second requires induction on $n<\omega$.
    The case where $n=0$ is trivial.
    For the inductive step, asssume
    \[
      \proves\rat{\min\{n(1-q),1\}}\leftrightarrow(q\narrow n\rat 0)
    \]
    for some $n<\omega$.
    By the first part, we see
    \[
      \proves\Big(\rat{\min\{n(1-q),1\}}\to(q\narrow n\rat 0)\Big)\to\Big(\rat{q\Rightarrow\min\{n(1-q),1\}}\to(\rat q\narrow{n+1}\rat 0))\Big).
    \]
    So we may use modus ponens and the inductive hypothesis to get
    \[
      \proves\rat{q\Rightarrow\min\{n(1-q),1\}}\to(\rat q\narrow{n+1}\rat 0).
    \]
    which gives us one of the directions because
    \[
      (q\Rightarrow\min\{n(1-q),1\})=\min\{(n+1)(1-q), 1\}.
    \]
    For the other direction, one first infers
    \[
      \proves(\rat q\narrow{n+1}\rat 0)\to(\rat q\to\rat{\min\{n(1-q),1\}})
    \]
    by the induction hypothesis and $\aref{rev-trans}$.
    From there, we use this proof:
    \[
      \infer[\trans]{(\rat q\narrow{n+1}\rat 0)\to\rat{\min\{(n+1)(1-q),1\}}}{
        \infer*{(\rat q\narrow{n+1}\rat 0)\to(\rat q\to\rat{\min\{n(1-q),1\}})}{} &
        \infer[\aref{rationals}]{(\rat q\to\rat{\min\{n(1-q),1\}})\to\rat{\min\{(n+1)(1-q),1\}}}{}
      }\qedhere
    \]
  \end{proof}
\end{lemma}
The lemma above allows us a quick proof that any rational can be used to prove 0, which is well-known for finitary {\luk}-Pavelka Logic.
\begin{lemma}[Sorites Paradox]
  Let $q$ be a rational in $\Q\cap[0,1]$.
  If $q<1$, then 
  \[
    \rat q\proves\rat 0\quad\text{and}\quad\proves\neg\strict{\rat q}.
  \]
  \begin{proof}
    Because $q<1$, we may choose $n<\omega$ large enough so that $n(1-q)\geq 1$.
    Therefore, $\min\{n(1-q),\ 1\}=1$.
    So by the lemma above (\ref{lem:rationals-basic}) and modus ponens, we have $\proves\rat q\narrow n\rat 0$.
    Hence, repeating modus ponens $n$ times will yield $\rat q\proves\rat 0$.
    
    By the Deduction Theorem, $\rat q\proves\rat 0$ implies $\proves\strict{\rat q}\to\rat 0$, which is just $\proves\neg\strict{\rat q}$ without the shorthand.
  \end{proof}
\end{lemma}

Having rationals prove $\rat 0$ provides a nice segue into inconsistency, defined below.
\begin{definition}
  A theory $T$ is \emph{inconsistent} if $T\proves\phi$ for all sentences $\phi$.
\end{definition}
\begin{lemma}\label{lem:alt-def-inconsistency}
  A theory $T$ is inconsistent iff $T\proves\rat q$ for some/all $q\in\Q\cap[0,1)$.
  \begin{proof}
    The left-to-right direction follows straight from the definition of inconsistency.
    
    We prove the opposite direction.
    Let $q\in\Q\cap[0,1)$ be given such that $T\proves\rat q$.
    We've shown that $\rat q\proves\rat 0$, so we know that $T\proves\rat 0$.
    Because $\rat 0$ implies everything (as shown by \ref{itm:axiom-0-arrow}), we know that $T$ is inconsistent.
  \end{proof}
\end{lemma}

\begin{lemma}[Prelinearity]\label{lem:prelinearity}
  Let $T$ be a theory and $\phi$ and $\psi$ be formulas.
  The following hold:
  \begin{itemize}
  \item $\proves\neg\strict{\phi\to\psi}\to(\psi\to\phi)$.
  \item If $T\cup\{\phi\to\psi\}$ is inconsistent, then $T\proves\psi\to\phi$.
  \item If $T$ is consistent, then at least one of $T\cup\{\phi\to\psi\}$ or $T\cup\{\psi\to\phi\}$ is consistent.
  \end{itemize}
  \begin{proof}
    By Lemma~\ref{lem:vee-with-strict}, we have
    \[
      (\phi\to\psi)\vee(\psi\to\phi)\proves\neg\strict{\phi\to\psi}\to(\psi\to\phi).
    \]
    Because $(\phi\to\psi)\vee(\psi\to\phi)$ is exactly \ref{itm:axiom-order}, we know
    \[
      \proves\neg\strict{\phi\to\psi}\to(\psi\to\phi).
    \]
    That concludes the first part.
    
    By Deduction, $T\cup\{\phi\to\psi\}$ is inconsistent if only if $T\proves\neg\strict{\phi\to\psi}$.
    From there, the first part implies $T\proves\psi\to\phi$.
    If $T\proves\psi\to\phi$ and $T$ is consistent, then by soundness we have $T\cup\{\psi\to\phi\}$ is consistent.
    That concludes the remaining parts.
  \end{proof}
\end{lemma}

\begin{definition}\label{def:proof-degree}
  Let $T$ be a theory and $\phi$ a formula.
  The \emph{Pavelka provability degree} of $\phi$ with respect to $T$, denoted $|\phi|_T$, is the value
  \[
    |\phi|_T:=\sup\{p\in\Q\cap[0,1]\where T\proves\rat p\to\phi\}.
  \]
  Note that $|\phi|_T$ is well defined (as $T$ will always proves $\rat 0\to\phi$) and may be irrational.
  The \emph{Pavelka truth degree} of $\phi$ with respect to $T$, denoted $\|\phi\|_T$, is the value
  \[
    \|\phi\|_T:=\inf\{\phi^\mathcal M\where \mathcal M\models T\}.
  \]
\end{definition}
\begin{remark}
In finitary logic, it may be possible that $|\phi|_T=1$ and yet $T\not\proves\phi$;
however, infinitary logic is more expressive and does not have this problem, as shown in the lemma below.
\end{remark}
\begin{lemma}\label{lem:approach}
  Let $T$ be a theory and $\phi$ a formula.
  If $|\phi|_T$ is rational, then
  \[
    T\proves\rat{|\phi|_T}\to\phi
  \]
  \begin{proof}
    Assume $|\phi|_T$ is rational and take $q:=|\phi|_T$.
    By definition of $|\phi|_T$, we know that $T\proves\rat{q_0}\to\phi$ for all $q_0\in\Q\cap[0,1]$ with $q_0<q$.
    By Lemma~\ref{lem:rationals-basic}, we have for all $q_0<q$:
    \[
      T\proves\rat{q\Rightarrow q_0}\to(\rat q\to\phi)
    \]
    Notice that $q\Rightarrow q_0$ approaches 1 as $q_0$ approaches $q$.
    Therefore, one may show that $T\proves\rat p\to(\rat q\to\phi)$ for all $p\in\Q\cap[0,1)$.
    We finish with this proof:
    \[
      \infer[\arr\elim]{\rat q\to\phi}{
        \infer[\aref{approach}]{\bigvee_{p\in\Q\cap[0,1)}\rat p}{} &
        \infer[\pushdown]{(\bigvee_{p\in\Q\cap[0,1)}\rat p)\to(\rat q\to\phi)}{
          \infer[\unwedge\intro]{\bigwedge_{p\in\Q\cap[0,1)}(\rat p\to(\rat q\to\phi))}{
            \infer*{\rat p\to(\rat q\to\phi)\text{ (for all $p\in\Q\cap[0,1)$)}}{T}
          }
        }
      }\qedhere
    \]
  \end{proof}
\end{lemma}

\begin{lemma}[Narrowing]\label{lem:narrowing}
  Let $\phi$ be a sentence and $T$ a consistent theory.
  Both of the following hold:
  \begin{itemize}
  \item For any rational $q\in\Q\cap[0,1]$ such that $q>|\phi|_T$, the theory $T\cup\{\phi\to\rat q\}$ is consistent.
  \item For any $\epsilon>0$, there is some $p,q\in\Q\cap[0,1]$ with $p\leq q<p+\epsilon$ such that $T\cup\{\rat p\to\phi, \phi\to\rat q\}$ is consistent.
  \end{itemize}
  \begin{proof}
    By definition of $|\phi|_T$, we know that $T\not\proves\rat q\to\phi$ for any $q\in\Q\cap[0,1]$ such that $q>|\phi|_T$.
    So by Lemma~\ref{lem:prelinearity}, we know that $T\cup\{\phi\to\rat q\}$ is consistent for any such $q\in\Q\cap[0,1]$.
    That completes the first claim.
    
    Now for the second claim.
    If $|\phi|_T=1$, then we can just take $p=q=1$.
    Otherwise, we take $p,q\in\Q\cap[0,1]$ such that
    \[
      |\phi|_T-\tfrac\epsilon2<p\leq|\phi|_T<q<|\phi|_T+\tfrac\epsilon2.
    \]
    By the first claim, we know that $T\cup\{\rat p\to\phi,\phi\to\rat q\}$ is consistent.
  \end{proof}
\end{lemma}
\begin{lemma}\label{lem:valid-constraint}
  For any consistent theory $T$, the set $s_T$ of constraints proven by $T$, ie.\ 
  \begin{align*}
    s_T:=&\{\rat p\to\phi\where\text{$\phi$ a sentence and $p\in\Q\cap[0,1]$ such that $T\proves\rat p\to\phi$}\}\\
    &\cup\{\phi\to\rat q\where\text{$\phi$ a sentence and $q\in\Q\cap[0,1]$ such that $T\proves\phi\to\rat q$}\},
  \end{align*}
  is a valid constraint-theory (recall Definition~\ref{def:constraint}).
  \begin{proof}
    Assume it weren't, then there is some sentence $\phi$ and rationals $p,q\in\Q\cap[0,1]$ such that $\rat p\to\phi$ and $\phi\to\rat q$ are in $s_T$ but $q<p$.
    By transitivity, we would have $T\proves\rat p\to\rat q$, but that gives $T\proves\rat{p\Rightarrow q}$, which is a rational less than 1 and so makes $T$ inconsistent.
  \end{proof}
\end{lemma}
\begin{lemma}\label{lem:proof-theoretic-consistency-property}
  Let $S$ be a signature containing a countable set of constants $C$ and let $\frag$ be a fragment of $\lang(S)$.
  Take $\mathcal T$ be the set of all consistent $\frag$-theories that reference only finitely many constants from $C$.
  Using the notation from the previous lemma, the set of constraint-theories defined by
  \[
    \{s\subseteq\frag\where\text{$s$ is countable and there is some $T\in\mathcal T$ such that $s\subseteq s_T$}\}
  \]
  is a consistency property.
  \begin{proof}
    Let $\mathscr{S}$ be the set of constraint-theories defined above.
    
    Let $s\in \mathscr{S}$ be given and take $T\in\mathcal T$ such that $s\subseteq s_T$.
    We've proven in Lemma~\ref{lem:valid-constraint} that $s_T$ is valid, so clearly $s$ must be as well.
    
    Let $p,q,r\in\Q\cap[0,1]$ be given.
    We now prove each rule separately:
    \begin{enumerate}[label=(CP\arabic*)]
    \item
      Let $\theta_1$ and $\theta_2$ be basic sentences.
      \begin{itemize}
      \item
        Assume that $\rat p\to\theta_1$ and $\theta_2\to\rat q$ are in $s$.
        One can show that this implies $T\proves(\theta_1\to\theta_2)\to(\rat p\to\rat q)$.
        Therefore,
        \[
          T\proves(\theta_1\to\theta_2)\to\rat{p\Rightarrow q}
        \]
        so if $\rat r\to(\theta_1\to\theta_2)$ is in $s$, then $r$ must be no more than $(p\Rightarrow q)$ to preserve validity.
      \item 
        Assume that $\theta_1\to\rat p$ and $\rat q\to\theta_2$ are in $s$.
        One can show that this implies $T\proves(\rat p\to\rat q)\to(\theta_1\to\theta_2)$.
        Therefore,
        \[
          T\proves\rat{p\Rightarrow q}\to(\theta_1\to\theta_2)
        \]
        so if $\rat r\to(\theta_1\to\theta_2)$ is in $s$, then $r$ must be no less than $(p\Rightarrow q)$ to preserve validity.
      \end{itemize}
	  \item
	    Let $\phi$ be a formula such that $\rat p\to\phi$ is in $s$ and let $\psi$ be an extended arrow-manipulation of $\phi$.
      One can show by induction on the complexity of formulas that $\proves\phi\leftrightarrow\psi$, so we know that $s\cup\{\rat p\to\phi\}$ is a subset of $s_T$ and thus is in $\mathscr{S}$.
    \item 
	    Let $\Phi$ be a countable set of sentences such that $\rat p\to\unwedge\Phi$ is in $s$.
	    For all $\phi\in\Phi$, we know $T\proves\rat p\to\phi$ because $T\proves\rat p\to\unwedge\Phi$, so the set $s\cup\{\rat p\to\phi\}$ is a subset of $s_T$ and is thus in $\mathscr{S}$.
	  \item
	    Let $\Phi$ be a countable set of sentences such that $\rat p\to\unvee\Phi$ is in $s$ and let $p_0\in(\Q\cap[0,p))\cup\{0\}$ be given.
	    If $p_0=0$, then we know $T\proves\rat{p_0}\to\phi$ for all $\phi\in\Phi$ so $s\cup\{\rat{p_0}\to\phi\}$ is in $s_T$ and thus in $\mathscr{S}$ for every $\phi\in\Phi$.
	    If $p_0\neq 0$, then assume for the sake of contradiction that $T\cup\{\rat{p_0}\to\phi\}$ is inconsistent for all $\phi\in\Phi$.
	    Lemma~\ref{lem:prelinearity} therefore says $T\proves\phi\to\rat{p_0}$ for all $\phi$, but that means we have the following proof:
	    \[
	      \infer[\pushdown]{(\unvee\Phi)\to\rat{p_0}}{
	        \infer[\unwedge\intro]{\bigwedge_{\phi\in\Phi}(\phi\to\rat{p_0})}{
	          \infer*{\phi\to\rat{p_0}\text{ (for all $\phi\in\Phi$)}}{T}
	        }
	      }
	    \]
	    However, by assumption we know that $T\proves p\to(\unvee\Phi)$ and $p_0<p$, so $T$ must be inconsistent.
	    That's a contradiction, so we can pick some $\phi\in\Phi$ such that $T\cup\{\rat{p_0}\to\phi\}$ is consistent. Notice that
	    \[
	      s\cup\{\rat{p_0}\to\phi\}\subseteq s_{T\cup\{\rat{p_0}\to\phi\}},
	    \]
	    and so $s\cup\{\rat{p_0}\to\phi\}\in \mathscr{S}$.
	  \item
	    Let $\phi(x)$ be a formula such that $\rat{p}\to\forall x\,\phi(x)$ is in $s$.
	    Then by Axiom~\ref{itm:axiom-forall-arrow}, we know that $T\proves\rat{p}\to\phi(t)$ for any term $t$.
	    Therefore, for any closed term $t$, the constraint-theory $s\cup\{\rat{p}\to\phi(t)\}$ is a subset of $s_T$ and so in $\mathscr{S}$.
	  \item
	    Let $\phi(x)$ be a formula such that $\rat p\to\exists x\,\phi(x)$ is in $s$ and let $p_0$ in $(\Q\cap[0,p))\cup\{0\}$ be given.
	    If $p_0=0$, then trivially we have $s\cup\{\rat 0\to\phi(t)\}\in \mathscr{S}$ for any closed term $t$.
	    
	    Assume that $p_0\neq 0$ and assume for the sake of contradiction that $T\cup\{\rat{p_0}\to\phi(t)\}$ is inconsistent for all closed terms $t$.
	    Let $c\in C$ be not in $T$.
	    We see that $T\proves\phi(c)\to\rat{p_0}$.
	    However, $T$ does not reference $c$, so we may replace $c$ with some variable $x$ to get $T\proves\phi(x)\to\rat{p_0}$.
	    Therefore,
	    \[
	      \infer[\pushdown]{(\exists x\,\phi(x))\to\rat{p_0}}{
	        \infer[\generalization]{\forall x(\phi(x)\to\rat{p_0})}{
	          \infer*{\phi(x)\to\rat{p_0}}{T}
	        }
	      }
	    \]
	    and that contradicts $T\proves\rat p\to\exists x\,\phi(x)$.
    \item
      Because $\proves\rat q\to\rat q$ is true, we know that $s\cup\{\rat q\to\rat q\}\subseteq s_T$ and thus is in $\mathscr{S}$.
    \item This rule is satisfied by using \ref{itm:axiom-0-arrow}.
	  \item This rule is a straightforward application of the Narrowing Lemma~(\ref{lem:narrowing}).
    \end{enumerate}
    We've shown that all rules hold, so $\mathscr{S}$ is a consistency property.
  \end{proof}
\end{lemma}
\begin{theorem}\label{thm:completeness}
  We have the following two results for any countable theory $T$ and sentence $\phi$.
  \begin{itemize}
  \item\emph{(Pavelka-Style Completeness)} $|\phi|_T=\|\phi\|_T$ \emph{(recall Definition~\ref{def:proof-degree})}.
  \item\emph{(Completeness)} $T\proves\phi$ if and only if $T\models\phi$.
  \end{itemize}
  \begin{proof}
    Completeness comes from Pavelka-style completeness and Lemma~\ref{lem:approach}, so we need only prove Pavelka-style completeness.
  
    Fix $T$ and $\phi$.
	
	  Soundness gives that $|\phi|_T\leq\|\phi\|_T$.
	  
	  The other direction of the inequality is trivial when $|\phi|_T=1$, so assume that $|\phi|_T<1$ and let $q\in\Q\cap[0,1]$ be given such that $q>|\phi|_T$.
	  By Lemma~\ref{lem:prelinearity}, we know $T\cup\{\phi\to\rat q\}$ is consistent.
	  Define the constraint-theory $s$ as
	  \[
	    s:=\{\rat 1\to\psi\where \psi\in T\}\cup\{\phi\to\rat q\}
	  \]
	  Because $T\cup\{\phi\to\rat q\}$ is consistent, we know that $s$ is in the consistency property defined in the lemma above (when taking the fragment to be any fragment containing $T\cup\{\phi\}$).
    Thus, by the model existence theorem, we know that there is some structure $\mathcal M$ realizing the set $\{\psi\where |\psi|^*_s=1\}$.
    This means that $\mathcal M\models T$ but $\mathcal M\models\phi\to\rat q$, hence $\phi^\mathcal M\leq q$.
    Therefore, $\|\phi\|_T\leq q$ for any $q$ above $|\phi|_T$, which means that $\|\phi\|_T\leq|\phi|_T$.
  \end{proof}
\end{theorem}

\section{Completeness for Fragments}
\newcommand{\caserule}{\text{\sc{Case}}}
This section attempts to generalize the results of the previous sections to proofs restricted to a fragment (recall Definition~\ref{def:fragment}).
We show that weak completeness holds in both the finite fragment and fragments that are closed under the floor formula.
We do not have the same result for general fragments, but there are many inference rules which, when added, give the desired result.

\begin{definition}
  Let $\frag$ be a fragment.
  For any $\frag$-theory $T$ and $\frag$-sentence $\phi$, we say
  \[
    T\proves_\frag\phi
  \]
  if there is a proof of $\phi$ assuming $T$ where every formula in the proof is in $\frag$.
  We define $|\phi|_T^\frag$ as
  \[
    |\phi|_T^\frag:=\sup\{p\in\Q\cap[0,1)\where T\proves_\frag \rat p\to\phi\}.
  \]
\end{definition}
\begin{definition}
  Let $\frag$ be a fragment.
  We say that \emph{Pavelka-style completeness holds} in $\frag$ if for all countable $\frag$-theories $T$ and $\frag$-sentence $\phi$, we have (recall Definition~\ref{def:proof-degree}):
  \[
    |\phi|_T^\frag=\|\phi\|_T
  \]
  We say that \emph{completeness holds} in $\frag$ if for all countable $\frag$-theories $T$ and $\frag$-sentence $\phi$, we have:
  \[
    T\proves_\frag\phi\text{ if and only if }T\models\phi
  \]
\end{definition}
\begin{theorem}
  If Pavelka-style completeness holds in a fragment that contains the formula 
  \[
    \bigvee_{p\in\Q\cap[0,1)}\rat p,
  \]
  then completeness holds in that fragment.
  \begin{proof}
    Fix a fragment $\frag$ such that weak completeness holds in $\frag$ and the formula $\bigvee_{p\in\Q\cap[0,1)}\rat p$ is in $\frag$.
    The first direction of completeness holds by soundness.
    Let $T$ be a countable $\frag$-theory and $\phi$ a $\frag$-sentence such that $T\models\phi$, meaning $\|\phi\|_T=1$.
    Weak completeness holds, so $|\phi|_T^\frag=1$.
    Notice that the proof in Lemma~\ref{lem:approach} stays within the fragment, so we may reuse it here to show that $T\proves\rat 1\to\phi$.
    The proof of $\rat 1$ is contained in every fragment, so we may apply modus ponens to get $T\proves_\frag\phi$.
  \end{proof}
\end{theorem}
\begin{definition}
  We say that a fragment is \emph{closed under the floor formula} if for every formula $\phi$ in the fragment, the formula $\strict{\phi}$ is in the fragment.
\end{definition}

\begin{theorem}
  Pavelka-style completeness holds in any fragment that is closed under the floor formula.
  Note that this theorem is superceded by Corollary~\ref{cor:floor-completeness}.
  \begin{proof}
    The proof for this is the same as Pavelka-style completeness in the previous section, the only difference being that the consistency property defined in Lemma~\ref{lem:proof-theoretic-consistency-property} needs to be restricted to the fragment in question.
    One can check that every proof from the last section works when restricted to a fragment, as long as that fragment is closed under the floor formula.
  \end{proof}
\end{theorem}

\begin{remark}
  The above two theorems would indicate that our definition of fragment should be expanded to always be closed under the floor formula and should always include the formula $\bigvee_{p\in\Q\cap[0,1)}\rat p$.
  However, we hesitate to add those restrictions because the former would stipulate that there are no \emph{continuous} fragments (a concept that is very important, and only defined, when a metric is added to the logic) and the latter prevents the set of finite sentences from being a fragment.
\end{remark}
\begin{remark}
  When a fragment is not closed under the floor formula, it has no associated deduction theorem.
  If one were to review the previous section, one would find that the deduction theorem was primarily used to show prelinearity: if a theory $T$ and sentence $\phi\to\psi$ are such that $T\cup\{\phi\to\psi\}\proves\rat 0$, then $T\proves\psi\to\phi$.
  This property is what allows us to show that if $T\not\proves\rat q\to\phi$, then there is some structure assigning $\phi$ a truth-value at most $q$.
  It is also what gives us the $\vee$-rule and $\exists$-rule, however those could be gotten simply by assuming the theory is globally consistent.

  Prelinearity is unfortunately metalogical.
  When trying to reestablish it, we are forced to add inference rules that involve assumptions that are closed upon invocation of the rule.
  The result is a more obnoxious proof-theory to induct over.
  
  However, before we add such an inference rule, we discuss the finite fragment, for which a deduction theorem is well known.
\end{remark}
\begin{theorem}[Deduction for Finite Proofs]\label{thm:deduction-finite}
  Let $\frag$ be a fragment.
  For any set of $\frag$-formulas $T$ and $\frag$-formulas $\phi$ and $\psi$, we have that $T\cup\{\phi\}\proves_\frag^{<\omega}\psi$ (meaning $T\cup\{\phi\}$ proves $\psi$ within $\frag$ in finitely many steps) iff there is some $n<\omega$ such that $T\proves_\frag^{<\omega}\phi\narrow{n}\psi$.
  \begin{proof}
      The converse direction is immediate.
      The proof for the forward direction is by induction on the lengths of proofs and is a quick corollary of Lemma~\ref{lem:no-contraction}.
  \end{proof}
\end{theorem}
\begin{lemma}[Prelinearity for Finite Proofs]\label{lem:prelinearity-finite}
  Let $\frag$ be fragment, $T$ a $\frag$-theory and $\phi$ and $\psi$ $\frag$-formulas.
  The following hold:
  \begin{itemize}
  \item $\proves^{<\omega}_\frag((\phi\to\psi)\narrow{n}\rat 0)\to(\psi\to\phi)$ for all $n<\omega$.
  \item If $T\cup\{\phi\to\psi\}\proves^{<\omega}_\frag\rat 0$, then $T\proves^{<\omega}_\frag\psi\to\phi$.
  \item If $T\not\proves^{<\omega}_\frag\rat 0$, then at least one of $T\cup\{\phi\to\psi\}\not\proves^{<\omega}_\frag\rat 0$ or $T\cup\{\psi\to\phi\}\not\proves^{<\omega}_\frag\rat 0$ holds.
  \end{itemize}
  \begin{proof}
    The first point comes from the first part of Lemma~\ref{lem:vee-with-strict} (which is allowable here because the proof was finite) and the Prelinearity Axiom~\ref{itm:axiom-order}.
    The remaining points are a corollary of the first and the Deduction Theorem for Finite Proofs~\ref{thm:deduction-finite}.
  \end{proof}
\end{lemma}
\begin{theorem}
  Pavelka-style completeness holds in the finite fragment.
  \begin{proof}
    Notice that every proof in the finite fragment is finite, or at least can be replaced with a finite proof.
    So by the previous lemma, we know that prelinearity holds in the fragment.
    That's enough to allow us to prove weak completeness for the finite fragment from the techniques used in last section's completeness result.
  \end{proof}
\end{theorem}
\begin{remark}
  As stated in the introduction, the previous theorem is very well known.
  However, we state it as above to demonstrate that it can be proven using this paper's model existence theorem.
\end{remark}
\begin{remark}
  As previously stated, we do not have an as-is completeness result for general fragments due to lacking a prelinearity result for such fragments.
  However, we can always add another inference rule that gives us prelinearity.
  Of course, we could add prelinearity directly via an inference rule like
  \[
    \infer{\psi\to\phi}{
      \infer*{\rat 0}{[\phi\to\psi]}
    }
  \]
  where the square brackets indicate an assumption that is made for this inference rule that is closed after the rule is invoked.
  In words, the above rule states ``if from the assumption of $\phi\to\psi$ one proves $\rat 0$, then one proves $\psi\to\phi$.''
  We could approach the problem more subtly with $\unvee$-Elimination ($\unvee\elim$):
  \[
    \infer{\chi}{
      \phi\vee\psi &
      \infer*{\chi}{[\phi]} &
      \infer*{\chi}{[\psi]}
    }
  \]
  Alternatively, we could do an infinite-casing type inference rule like one of these two:
  \begin{align*}
    \infer{\psi}{
      (\phi\narrow n\rat 0)\to\psi\text{ (for all $n<\omega$)} &
      \infer*{\psi}{[\phi]}
    } &&
    \infer{\psi}{
      \infer*{\psi}{[\phi]} &
      \infer*{\psi}{[\phi\to\rat q]}
      \text{(for all $q\in\Q\cap[0,1)$)}
    }
  \end{align*}
  All of these inference rules are valid, but some are better than others.
  The first rule in particular does not mesh well with the rest of the proof system, as evidenced by how difficult it is to prove the deduction theorem given that rule is in the system.
  The $\unvee$-Elimination rule appears in natural deduction, so it is in someway more usual to include it.
  The infinitary-casing rules actually give full completeness in all fragments, which is useful.
  While the casing on rationals is at first more natural, the first casing rule is actually more  closely related to our previous work, so we will discuss that one over the rational one.
  We will call said rule the \emph{case rule} ($\caserule$).
\end{remark}
\begin{theorem}\label{thm:redundancy}
  The $\unvee\elim$ and $\caserule$ inference rules are redundant in any fragment closed under the floor formula. 
  \begin{proof}
    Fix a fragment $\frag$ that is closed under the floor formula.
    
    The proof goes by showing that each invocation of the rules can be replaced to not use the rule.
    
    Let $T$ be a $\frag$-theory and $\phi,\psi,\chi$ $\frag$-formulas such that $T\proves\chi$ is inferred through one invocation of $\unvee\elim$, namely:
    \[
      \infer[\unvee\elim]{\chi}{
	      \phi\vee\psi &
	      \infer*{\chi}{[\phi]} &
	      \infer*{\chi}{[\psi]}
	    }
    \]
    with the proof above such that the proof from $\phi$ to $\chi$ and from $\psi$ to $\chi$ does not invoke $\unvee\elim$.
    We use deduction to see that $T\proves\strict\phi\to\chi$ and $T\proves\strict\psi\to\chi$.
    From there, we use this proof:
    \[
      \infer[\arr\elim]{\chi}{
        \infer=[\text{Lemma~\ref{thm:delta-strict}}]{\strict\phi\vee\strict\psi}{\phi\vee\psi}&
        \infer[\pushdown]{(\strict\phi\vee\strict\psi)\to\chi}{
          \infer[\unwedge\intro]{(\strict\phi\to\chi)\wedge(\strict\psi\to\chi)}{
            \strict\phi\to\chi &
            \strict\psi\to\chi
          }
        }
      }
    \]
    So we can replace any instance of $\unvee\elim$ in a proof.
    
    We now move onto $\caserule$.
    Let $T$ be a $\frag$-theory and $\phi,\psi$ $\frag$-formulas such that $T\proves\psi$ is inferred through one invocation of $\caserule$, namely:
    \[
      \infer[\caserule]{\psi}{
	      (\phi\narrow n\rat 0)\to\psi\text{ (for all $n<\omega$)} &
	      \infer*{\psi}{[\phi]}
	    } 
    \]
    By deduction, we have $T\proves\strict\phi\to\psi$.
    The fact that $T\proves(\phi\narrow{n}\rat 0)\to\psi$ gives:
    \[
	    \infer[\trans]{\neg\strict\phi\to\psi}{
        \infer[\aref{pushdown}]{\neg\strict\phi\to\bigvee_{n<\omega}\neg\neg(\phi\narrow{n}\rat 0)}{} &
        \!\!\!\!\!\!\!\!\!\!\!\!\!\!\!\!\!\!\!\!\!\!\!\!\!\!\!\!\!\!\!\!\!\!\!\!\!\!\!\!\!\!\!
        \infer[\pushdown]{\left(\bigvee_{n<\omega}\neg\neg(\phi\narrow{n}\rat 0)\right)\to\psi}{
          \infer[\unwedge\intro]{\bigwedge_{n<\omega}(\neg\neg(\phi\narrow{n}\rat 0)\to\psi)}{
            \infer[\trans]{\neg\neg(\phi\narrow{n}\rat 0)\to\psi\text{ (for all $n<\omega$)}}{
              \infer[\aref{classic}]{\neg\neg(\phi\narrow{n}\rat 0)\to(\phi\narrow{n}\rat 0)}{} &
              (\phi\narrow{n}\rat 0)\to\psi\text{ (for all $n<\omega$)}
            }
          }
        }
      }
    \]
    So we finish with this proof:
    \[
      \infer[\arr\elim]{\psi}{
        \infer[\aref{loe}]{\strict\phi\vee\neg\strict\phi}{} &
        \infer[\pushdown]{(\strict\phi\vee\neg\strict\phi)\to\psi}{
          \infer[\unwedge\intro]{(\strict\phi\to\psi)\wedge(\neg\strict\phi\to\psi)}{
            \infer*{\strict\phi\to\psi}{T} &
            \infer*{\neg\strict\phi\to\psi}{T}
          }
        }
      }
    \]
    That concludes this proof.
  \end{proof}
\end{theorem}
\begin{theorem}\label{thm:infer-completeness}
  Weak completeness holds in any fragment when the $\unvee\elim$ rule is allowed.
  \begin{proof}
    This proof is again the same as the other weak completeness proofs, so we only establish prelinearity here and leave the rest to the reader.
    Let $\frag$ be any fragment, $T$ be a countable $\frag$-theory, and $\phi,\psi$ formulas such that
    \[
      T\cup\{\phi\to\psi\}\proves_\frag\rat 0.
    \]
    We may then invoke this proof:
    \[
      \infer[\unvee\elim]{\psi\to\phi}{
        \infer=[\aref{order}]{(\phi\to\psi)\vee(\psi\to\phi)}{} &
        \infer[\arr\elim]{\psi\to\phi}{
          \infer*{\rat 0}{T,\ [\phi\to\psi]} &
          \infer=[\aref{0-arrow}]{\rat 0\to(\psi\to\phi)}{}
        } &
        [\psi\to\phi]
      }
    \]
    Thus, prelinearity is established.
  \end{proof}
\end{theorem}
\begin{theorem}\label{thm:completeness-in-fragments}
  Completeness holds in any fragment when the $\caserule$ rule is allowed.
  \begin{proof}
    As before, the original proof of weak completeness works for this case once one has prelinearity.
    Let $\frag$ be any fragment, $T$ be a countable $\frag$-theory, and $\phi,\psi$ formulas such that 
    \[
      T\cup\{\phi\to\psi\}\proves_\frag\rat 0.
    \]
    We may then invoke this proof:
    \[
      \infer[\caserule]{\psi\to\phi}{
        \infer=[\text{Lemma~\ref{lem:prelinearity}}]{((\phi\to\psi)\narrow{n}\rat 0)\to(\psi\to\phi)\text{ (for all $n<\omega$)}}{
          \infer=[\aref{order}]{(\phi\to\psi)\vee(\psi\to\phi)}{}
        } &
        \!\!\!\!\!\!\!\!\!\!
        \infer[\arr\elim]{\psi\to\phi}{
          \infer*{\rat 0}{T,\ [\phi\to\psi]} &
          \infer=[\aref{0-arrow}]{\rat 0\to(\psi\to\phi)}{}
        } &
      }
    \]
    That gives prelinearity and thus weak completeness.
    
    We must now establish completeness.
    Let $\frag$ be any fragment, $T$ a $\frag$-theory, and $\phi$ a $\frag$-formula such that $\|\phi\|_T=1$.
    We thus know $|\phi|_T^\frag=1$, so $T\proves_\frag\rat p\to\phi$ for each $p\in\Q\cap[0,1)$.
    
    Consider any $n<\omega$.
    We can take some rational $p\in\Q\cap[0,1)$ large enough so that $n(1-p)<1$.
    Therefore, we have the following proof:
    \[
      \infer[\trans]{(\phi\narrow{n}\rat 0)\to\phi}{
        \infer=[\text{Lemma~\ref{lem:proof-technical}}]{(\phi\narrow{n}\rat 0)\to(\rat p\narrow n\rat 0)}{
          \infer*{\rat p\to\phi}{}
        } &
        \infer[\trans]{(\rat p\narrow n\rat 0)\to\phi}{
          \infer=[\text{Lemma~\ref{lem:rationals-basic}}]{(\rat p\narrow n\rat 0)\to\rat{n(1-p)}}{} &
          \infer*{\rat{n(1-p)}\to\phi}{}
        }
      }
    \]
    
    The above $n$ was arbitrary, so we prove $\phi$ by $\caserule$ as follows:
    \[
      \infer[\caserule]{\phi}{
        \infer*{(\phi\narrow n\rat 0)\to\phi\text{ (for all $n<\omega$)}}{} &
        [\phi]
      }
    \]
    We have thus established completeness from weak completeness.
  \end{proof}
\end{theorem}
\begin{corollary}\label{cor:floor-completeness}
  Completeness holds in any fragment that is closed under the floor formula.
  \begin{proof}
    Completeness holds for any fragment when adding $\caserule$ to the logic system's inference rules and $\caserule$ is redundant when the fragment is closed under the floor formula.
  \end{proof}
\end{corollary}
\section{In Relation to Fuzzy Logic}\label{sec:fuzzy-logic}
This section relates the work above with that of fuzzy logicians.

In this section, we show that the floor formula acts as a derived Baaz Delta, which we define in Definition~\ref{def:baaz}.
The Baaz Delta was introduced in~\cite{baaz1996infinite} to study G\"odel logic (another type of fuzzy logic), though~\cite{rose1958fragments} references the same operator, simply called `J' in the paper, for use in {\L}ukasiewicz logic in the 1950s.

Before talking about the Baaz Delta, let us take a quick aside to mention some notation.
In Remark~\ref{rem:issues}, we stated that fuzzy logicians often use an additional connective.
We introduce it now.
\begin{definition}
  We add to the language the derived binary connective `$\odot$,' called the \emph{{\L}ukasiewicz t-norm}, defined so that for all formulas $\phi$ and $\psi$, the formula $\phi\odot\psi$ is
  \[
    \neg(\phi\to\neg\psi).
  \]
  In addition, for any formula $\phi$ and $n<\omega$, we define the formula\footnote{The formula $\phi^n$ is endearingly referred to in~\cite{metcalfe2008proof} as a \emph{confusion} of $\phi$.} $\phi^n$ to be
  \[
    \underbrace{\phi\odot\phi\odot\cdots\odot\phi\odot\phi}_\text{$n$ times}
  \]
\end{definition}
\begin{remark}
  Fix formulas $\phi$, $\psi$, and $\chi$.
  For any structure $\mathcal M$, the formula $\phi\odot\psi$ has truth-value:
  \[
    \max\{\phi^\mathcal M+\psi^\mathcal M-1, 0\}
  \]
  With basic invoking of definitions, we see:
  \[
    \phi\to(\psi\to\chi)\equiv (\phi\odot\psi)\to\chi
  \]
  So that for any $n<\omega$, the formula $\phi\narrow{n}\psi$ is equivalent to $\phi^n\to\psi$.
  Therefore:
  \[
    \strict{\phi}\equiv \bigwedge_{n<\omega}\phi^n
  \]
\end{remark}
\begin{definition}\label{def:baaz}
  We define the logic $\baselang_\triangle$ by extending the logic $\lang$ defined in this paper with a new unary operator `$\triangle$,' called the \emph{Baaz Delta}, which is interpreted semantically for any structure $\mathcal M$ and formula $\phi$ as:
  \[
    [\triangle\phi]^\mathcal M:=\left\{\begin{array}{ll} 
      1&\text{if $\phi^\mathcal M=1$}\\
      0&\text{otherwise}
    \end{array}\right.
  \]
  In addition, we add the inference rule:
  \[
    \infer{\triangle\phi}{\phi}
  \]
  and the axiom schemas:
  \begin{enumerate}[label=(A$\triangle$\arabic*)]
  \item\label{itm:axiom-loe-baaz} $\triangle\phi\vee\neg\triangle\phi$
  \item\label{itm:axiom-vee-baaz} $\triangle(\phi\vee\psi)\to(\triangle\phi\vee\triangle\psi)$
  \item\label{itm:axiom-strict-baaz} $\triangle\phi\to\phi$
  \item\label{itm:axiom-double-strict-baaz} $\triangle\phi\to\triangle\triangle\phi$
  \item\label{itm:axiom-impl-baaz} $\triangle(\phi\to\psi)\to(\triangle\phi\to\triangle\psi)$
  \end{enumerate}
\end{definition}
\begin{theorem}\label{thm:delta-strict}
  The new axioms and inference rules of $\baselang_\triangle$ are derivable in $\lang$ when replacing references of $\triangle\phi$ with $\strict\phi$.
  \begin{proof}
    The inference rule was shown to hold in Lemma~\ref{lem:proves-strict}.
    The law of the excluded middle~\ref{itm:axiom-loe-baaz} was taken as axiom~\ref{itm:axiom-loe} in our logic, though it's interesting to note that the semantic proof of $\strict{\phi}\vee\neg\strict\phi$ has more to do with the Archimedean principle than with logic.
    We showed~\ref{itm:axiom-strict-baaz} with \ref{itm:axiom-strict-arrow}.
    The axioms~\ref{itm:axiom-double-strict-baaz} and~\ref{itm:axiom-impl-baaz} come from the Deduction Theorem~\ref{thm:deduction} and Lemma~\ref{lem:proves-strict} as follows:
    \begin{align*}
      \phi&\proves\phi&  \phi\to\psi,\phi&\proves\psi\\
      \phi&\proves\strict\phi&  \phi\to\psi,\phi&\proves\strict\psi\\
      \phi&\proves\strict{\strict\phi}&  \phi\to\psi&\proves\strict\phi\to\strict\psi\\
      &\proves\strict{\phi}\to\strict{\strict\phi}&
      &\proves\strict{\phi\to\psi}\to(\strict\phi\to\strict\psi)
    \end{align*}
    The only remaining axiom to derive is~\ref{itm:axiom-vee-baaz}.
    By the Deduction Theorem, we need only show $\phi\vee\psi\proves\strict\phi\vee\strict\psi$.
    First we prove that $\phi\vee\psi\proves\strict\phi\vee\psi$ as follows:
    \[
      \infer[\arr\elim]{\strict\phi\vee\psi}{
        \infer[\aref{loe}]{\strict\phi\vee\neg\strict\phi}{} &
        \infer[\pushdown]{(\strict\phi\vee\neg\strict\phi)\to(\strict\phi\vee\psi)}{
          \infer[\unwedge\intro]{(\strict\phi\to(\strict\phi\vee\psi))\wedge(\neg\strict\phi\to(\strict\phi\vee\psi)}{
            \infer=[\aref{vee-arrow}]{\strict\phi\to(\strict\phi\vee\psi)}{} &
            \infer[\trans]{\neg\strict\phi\to(\strict\phi\vee\psi)}{
              \infer=[\text{Lemma~\ref{lem:vee-with-strict}}]{\neg\strict\phi\to\psi}{\phi\vee\psi} &
              \infer=[\aref{vee-arrow}]{\psi\to(\strict\phi\vee\psi)}{}
            }
          }
        }
      }
    \]
    The same proof gives $\strict\phi\vee\psi\proves\strict\phi\vee\strict\psi$, so we have $\phi\vee\psi\proves\strict\phi\vee\strict\psi$.
  \end{proof}
\end{theorem}
\begin{remark}
  Though this remark is trivial, it should be stated that there are countable sets $\Phi$ such that:
  \[
    \bigvee_{\phi\in\Phi}\phi\not\models\bigvee_{\phi\in\Phi}\strict\phi
  \]
  Specifically, consider $\bigvee_{p\in\Q\cap[0,1)}\rat p$, which has valuation 1 yet the floor of each formula is 0.
  This explains why the proof of $\phi\vee\psi\proves\strict\phi\vee\strict\psi$ was more difficult than the other parts of the previous theorem.
\end{remark}
\begin{remark}
  Baaz and Metcalfe~\cite{baaz2007lukasiewicz} proved completeness for {\luk} logic by adding the infinitary inference rule
  \[
    \infer{\phi}{\phi\oplus\phi^n\text{ (for all $n<\omega$)}}
  \]
  where $\oplus$ is semantically the sum of its components (truncated at $1$).
  We did not use this rule in the previous section because it is useful only for proving completeness from Pavelka-style completeness, rather than showing Pavelka-style completeness.
  However, it's worth mentioning that the work in this paper gives a little extra intuition on Baaz's rule.
  Semantically, the rule is equivalent to:
  \[
    \infer{\phi}{(\phi\narrow{n}\rat 0)\to\phi\text{ (for all $n<\omega$)}}
  \]
  Essentially, this means Baaz's rule is a way to prove $\neg\strict\phi\to\phi$ because
  \[
    \infer[\pushdown]{(\bigvee_{n<\omega}(\phi\narrow{n}\rat 0))\to\phi}{
      \infer[\unwedge\intro]{\bigwedge_{n<\omega}((\phi\narrow{n}\rat 0)\to\phi)}{
        (\phi\narrow{n}\rat 0)\to\phi\text{ (for all $n<\omega$)}
      }
    }
  \]
  and one can show
  \[
    \neg\strict\phi\leftrightarrow\Big(\neg\bigwedge_{n<\omega}\neg(\phi\narrow n\rat0)\Big)\leftrightarrow\Big(\neg\neg\bigvee_{n<\omega}(\phi\narrow n\rat 0)\Big)\leftrightarrow\bigvee_{n<\omega}(\phi\narrow n\rat 0).
  \]
  Of course, $\phi$ can be proven from knowing $\neg\strict\phi\to\phi$ via:
  \[
    \infer[\arr\elim]{\phi}{
      \infer[\aref{loe}]{\strict\phi\vee\neg\strict\phi}{} &
      \infer[\pushdown]{(\strict\phi\vee\neg\strict\phi)\to\phi}{
        \infer[\unwedge\intro]{(\strict\phi\to\phi)\wedge(\neg\strict\phi\to\phi)}{
          \infer=[\aref{strict-arrow}]{\strict\phi\to\phi}{} &
          \neg\strict\phi\to\phi
        }
      }
    }
  \]
\end{remark}


\section{Downward L\"owenheim-Skolem-Tarski and Indiscernibles}
Throughout this section, we fix a countable signature $S$ and a countable fragment of $\lang(S)$ denoted $\frag$.
All structures we refer to will be $S$-structures unless stated otherwise.

This section is an adaptation of chapter 13 of~\cite{keisler1971model}.
Uncontroversially, we may state that Skolem functions are useful.
Unfortunately, they are often realized via the axiom of choice, which prevents us from forcing them to be continuous (when a metric is in the logic).
It is for this reason alone that we removed the metric from our paper.
Fortunately, Skolem functions often behave as scaffolding, meaning they are not referenced in either the statement of a theorem (ie.\ they are removed in order to reach the conclusion); this means we can get many results for continuous logic by selectively removing the continuity condition for Skolem functions and then removing those functions before reaching the conclusion.

This section is really just groundwork and notation.
It adapts indiscernibles and substructures from standard logic over to $[0,1]$-valued.
This section in particular should be straightforward for those who have studied model theory, so it is kept relatively terse (when compared to the previous sections).

\begin{definition}
  Let $\mathcal M$ and $\mathcal N$ be structures.
  We define $\mathcal M\equiv_\frag\mathcal N$ to mean that
  \[
    \mathcal M\models\phi \quad\text{if and only if}\quad \mathcal N\models\phi
  \]
  for all $\frag$-sentences $\phi$.
  We take $\mathcal M\preceq_\mathcal L\mathcal N$ to mean that $M\subseteq N$ and for all $\mathcal L$-formulas $\phi(\V x)$ and tuples $\V a\in M^{\len(\V x)}$ we have
  \[
    \mathcal M\models\phi[\V a]\quad\text{if and only if}\quad\mathcal N\models\phi[\V a].
  \]
\end{definition}

\begin{remark}
  If we have two structures $\mathcal M$ and $\mathcal N$ such that $\mathcal M\preceq_\frag\mathcal N$, then $\phi^\mathcal M(\V a)=\phi^\mathcal N(\V a)$ for all $\frag$-formulas $\phi(\V x)$ and tuples $\V a\in\mathcal M^{\len(\V x)}$.
\end{remark}

\begin{theorem}[Tarski-Vaught Test]\label{lem:tarski-vaught-test}
  Let $\mathcal M$ and $\mathcal N$ be two structures with $\mathcal M\subseteq\mathcal N$.
  The following are equivalent:
  \begin{itemize}
  \item $\mathcal M\preceq_\frag\mathcal N$
  \item 
     For every $\mathcal L$-formula $\phi(x;\V y)$ and every tuple $\V b\in M^{\len(\V y)}$, we have
     \[
       \mathcal N\models\exists x\,\phi(x;\V b)
     \]
     only if for every $p\in\Q\cap[0,1)$ there is a value $a\in M$ such that
     \[
       \mathcal M\models\rat p\to\phi[a;\V b].
     \]
  \end{itemize}
  \begin{proof}
    Induction on the complexity of formulas, showing that all valuations are necessarily the same.
  \end{proof}
\end{theorem}

\begin{definition}\label{def:skolem}
  For every rational $p\in[0,1)$ and $\frag$-formula of the form $\exists x\,\phi(x;\V y)$ with $\len(\V y)<\omega$, we introduce the Skolem function symbol $\skolem p{\phi}$ with arity $\len(\V y)$.
  We let $\fragsk$ denote the smallest fragment extending $\frag$ defined in an expanded signature such that there is a Skolem function symbol for every formula $\exists x\,\phi(x;\V y)$ in $\fragsk$.
  We define the Skolem-theory $\Tskolem$ as
  \[
    \Tskolem:=\Big\{(\forall\V y)\,\rat p\to\Big(\exists x\,\phi(x;\V y)\to\phi(\skolem p{\phi}(\V y);\V y)\Big)\where \phi\in\fragsk\text{ and }p\in\Q\cap[0,1)\Big\}.
  \]
\end{definition}

\begin{lemma}
  Every $S$-structure has an expansion into $\fragsk$ that models $\Tskolem$.
  \begin{proof}
    Straightforward.
  \end{proof}
\end{lemma}

\begin{lemma}
  Let $\mathcal N$ be a structure that satisfies $\Tskolem$.
  Any submodel of $\mathcal N$ is a $\fragsk$-elementary substructure of $\mathcal N$.
  \begin{proof}
    Just as in classical $\lang$, this proof follows from the Tarski-Vaught test.
    Let $\mathcal M$ be a submodel of $\mathcal N$.
    Fix any $\fragsk$-formula $\phi(x;\V y)$ with finitely many free variables and let $\V b\in M^{\len(\V y)}$ be given such that $\mathcal N\models\exists x\,\phi(x;\V b)$.
    
    Fix an arbitrary $p\in\Q\cap[0,1)$ and take $a:=\skolem p{\phi}^\mathcal N(\V b)$.
    The way $a$ is defined and the fact that $\mathcal M\subseteq\mathcal N$ means that $a\in M$.
    Additionally, we assumed $\mathcal N\models \Tskolem$, which implies $\mathcal N\models \rat p\to\phi[a;\V b]$.
    Thus, we may invoke the Tarski-Vaught test.
  \end{proof}
\end{lemma}

\begin{definition}
  For any structure $\mathcal M$ and set $A\subseteq M$, we define $\hull(A)$ to be the closure of $A$ under terms in the language of $\mathcal M$.
  Note that $\hull(A)$ will necessarily be a structure.
\end{definition}

\begin{theorem}[Downward L\"owenheim-Skolem-Tarski]
  For any infinite cardinal $\alpha$, structure $\mathcal N$, and set $A\subseteq N$ with $|A|\leq\alpha\leq|N|$, there exists a structure $\mathcal M$ satisfying $|M|=\alpha$, $A\subseteq M$, and $\mathcal M\preceq_\frag\mathcal N$.
  
  \begin{proof}
    The proof is the same as in classical logic: expand $\mathcal N$ to satisfy $\Tskolem$ and take $\mathcal M$ to be $\hull(A)$ reducted back to $\frag$.
  \end{proof}
\end{theorem}

\begin{notation}
  When referring to a linearly ordered set $\langle A,<\rangle$, we define the set $[A]^n$ for any $n<\omega$ as the set of all strictly increasing sequences in $A$ of length $n$.
\end{notation}

\begin{definition}
  Consider any structure $\mathcal M$.
  We define a \emph{sequence of indiscernibles} in $\mathcal M$ to be a linear-ordered set $\langle A,<\rangle$ satisfying $A\subseteq M$ and for any $n<\omega$ and ordered tuples $\V a,\V b\in[A]^n$, we have
  \[
    (\mathcal M,\V a)\equiv_\frag (\mathcal M,\V b).
  \]
  meaning that for all $\frag$-formulas $\phi(\V x)$ with $\len(\V x)=n$, we have
  \[
    \mathcal M\models\phi[\V a]\quad\text{if and only if}\quad\mathcal M\models\phi[\V b].
  \]
\end{definition}

\begin{theorem}[Stretching Theorem]\label{thm:stretching}
  Let $\mathcal M$ be a model of $\Tskolem$ and let $\langle A,<\rangle$ be an infinite sequence of indiscernibles in $\mathcal M$.
  For any infinite linear-ordered set $\langle B,<\rangle$, there exists a model $\mathcal N$ such that $\langle B,<\rangle$ is a sequence of indiscernibles in $\mathcal N$ and for all $n<\omega$ and tuples $\V a\in[A]^n$ and $\V b\in[B]^n$ we have:
  \[
    (\mathcal M,\V a)\equiv_\frag (\mathcal N,\V b)
  \]
  Note that because $n$ could be 0, we see this immediately implies $\mathcal M\equiv_\frag\mathcal N$.
  \begin{proof}
    Without loss of generality, we may assume that $\mathcal M$ is $\hull(A)$.
    Let an infinite linear-ordered set $\langle B,<\rangle$ be given.
    
    Define the set $\mathcal T$ of formal terms
    \[
      \mathcal T:=\{t[\V b]\where \text{$t$ is a term in $\fragsk$ and $\V b$ is an ordered tuple in $B$}\}.
    \]
    We proceed by defining an equivalence relation $\sim$ on $T$.
    Let $t_1[\V b_1]$ and $t_2[\V b_2]$ be elements of $T$; we say that $t_1[\V b_1]\sim t_2[\V b_2]$ iff 
    \[
      t_1^\mathcal M(\V a_1)=t_2^\mathcal M(\V a_2)\text{ for all $\V a_1\in[A]^{\len(\V b_1)}$ and $\V a_2\in[A]^{\len(\V b_2)}$}
    \]
    Define $N:=\mathcal T/{\sim}$.
    
    Because a single free-variable $x$ is a term, we know that $B\subseteq \mathcal T$.
    Of course, $a_1\neq a_2$ for any distinct $a_1,a_2\in A$, implying that $b_1\not\sim b_2$ for any distinct $b_1,b_2\in B$.
    Thus, we may unambiguously identify each $b\in B$ with its equivalence class in $N$.
    Therefore, $B\subseteq N$.
    
    Let $f$ be a function symbol in $\fragsk$ with some arity $n<\omega$.
    Let $u_1,\dots,u_n$ be in $N$, meaning we can take terms $t_1,\dots,t_n$ and ordered tuples $\V b_1,\dots,\V b_n$ in $B$ such that $u_i$ is the equivalence class of $t_i[\V b_i]$ over $\sim$ for each $i$ in 1 through $n$.
    Let $\V b$ be the ordered tuple containing all of $\V b_1,\dots,\V b_n$ and let $t$ be a term such that unpacks $\V b$ into the term $f(t_1[\V b_1],\dots,t_n[\V b_n])$, meaning that the equation
    \[
      t[\V b]=f(t_1[\V b_1],\dots,t_n[\V b_n])
    \]
    is established syntactically.
    We define
    \[
      f^\mathcal N(u_1,\dots, u_n):=t[\V b]/{\sim}.
    \]
    One can check that this definition is unambiguous.
    
    Consider any relation symbol $R$ of some arity $n<\omega$.
    Taking $u_1,\dots,u_n$, $t_1,\dots,t_n$, and $\V b_1,\dots,\V b_n$ as before,
    taking arbitrary ordered tuples $\V a_1,\dots,\V a_n$ from $A$ such that $\len(\V a_i)=\len(\V b_i)$ for all $i$, we define
    \[
      R^\mathcal N(u_1,\dots,u_n):=R^\mathcal M(t_1^\mathcal M(\V a_1),\dots,t_n^\mathcal M(\V a_n)).
    \]
    Again, one may check that $R^\mathcal N$ is unambiguously defined.
    
    Of course, one defines $c^\mathcal N$ for each constant $c$ in $\fragsk$ as the equivalence class of $c$ in $N$.
    
    From all this, we may define $\mathcal N:=(N, c^\mathcal N, f^\mathcal N,R^\mathcal N)_{c,f,R\in\fragsk}$ to achieve a structure as desired.
  \end{proof}
\end{theorem}

\section{Hanf Number}

In this section, we will prove that the Hanf number of infinitary $[0,1]$-valued logic is $\beth_{\omega_1}$.
This section is an adaptation of chapter 15 in~\cite{keisler1971model}.

The fact that the Hanf number exists and is $\beth_{\omega_1}$ (the same as for standard $\lang$) was pretty much expected, though it is of course good to have a proof.
The importance of this section is really in demonstrating how simple it is to translate a consistency property based model-theoretic proof from standard logic into one for $[0,1]$-valued logic using the work of this paper.

\begin{remark}
  The definition and theorem below are well known.
  For a proof of Ed\"os-Rado, see chapter 14 in~\cite{keisler1971model}.
\end{remark}
\begin{definition}
  For cardinals $\lambda,\mu,\kappa$ and positive natural $n<\omega$, we write
  \[
    \lambda\to(\mu)_\kappa^n
  \]
  to mean that for any ordered set $\langle A,<\rangle$ of size $\lambda$, set $B$ of size $\kappa$, and function $f:[A]^n\to B$, there is an element of $b\in B$ for which the inverse image $f^{-1}(b)$ has size $\kappa$.
\end{definition}
\begin{theorem}[Erd\"os-Rado]\label{thm:erdos-rado}
  For any infinite cardinal $\lambda$ and natural number $n<\omega$, we have:
  \[
    \beth_n(\lambda)^+\rightarrow(\lambda^+)^{n+1}_\lambda
  \]
\end{theorem}
\begin{corollary}\label{cor:erdos-rado-omega}
  For any $\alpha<\omega_1$ and $n<\omega$, we have 
  \[
    \beth_{\alpha+n}\rightarrow(\beth_\alpha)^n_{\aleph_0}.
  \]
  \begin{proof}
    Erd\"os-Rado gives $\beth_{n-1}(\beth_\alpha)^+\rightarrow(\beth_\alpha^+)^n_{\beth_\alpha}$, which is a stronger statement than we need to prove because $\beth_\alpha\geq\aleph_0$ and $\beth_{n-1}(\beth_\alpha)^+\leq\beth_{\alpha+n}$.
  \end{proof}
\end{corollary}
\begin{theorem}\label{thm:hanf}
  Let $T$ be an $\frag$-theory.
  If $T$ has a model of size $\beth_\alpha$ for all infinite $\alpha<\omega_1$, then:
  \begin{itemize}
  \item $T$ has a model with an infinite sequence of indiscernibles.
  \item $T$ has a model for all infinite powers.
  \end{itemize}
  \begin{proof}
    The fact that the first result implies the second is an easy consequence of the Stretching Theorem~(\ref{thm:stretching}).
    So we must only proof the first result.
    The proof is conceptually identical to the proof for classical logic.
    
    Expand the fragment $\frag$ to a fragment $\frag^*$ with a new binary relation symbol $\sim$ and two new countable set of constants $K$ and $C=\{c_1,c_2,\dots\}$.
    Define the theory $I$ as the set of all $\frag^*$-sentences of the form:
    \[
      \psi(c_{i_1},\dots,c_{i_n})\leftrightarrow \psi(c_{j_1},\dots,c_{j_n})\text{ where }\left\{\begin{array}{l}i_1<\cdots<i_n\\j_1<\cdots<j_n\\\psi(x_1,\dots,x_n)\in\frag\end{array}\right.
    \]
    Define the $\frag^*$-theory $E$ by
    \[
      E:=\{(\forall x)\,x\sim x\}\cup\{\neg(c_i\sim c_j) \mid i\neq j\}
    \]
    One can see that $E$ specifies each of the elements of $C$ are represented by distinct elements and that $I$ stipulates the set $C$ (ordered by index) forms a sequence of indiscernibles.
    Therefore, any model of $T\cup I\cup E$ reducted to $\frag$ will be a model of $T$ with an infinite set of indiscernibles.
    So we need only find a model of $T\cup I\cup E$, which we accomplish with the Extended Model Existence Theorem~(\ref{thm:model-existence-ex}), which requires us to define a consistency property.
    
    Take the set $\mathscr{S}$ to contain all finite $\frag^*$-constraint-theories $s$ such that:
    \begin{itemize}
    \item Only finitely many of $C$ and $K$ are referenced in $s$, which we denote $\V c_s$ and $\V k_s$ respectively.
    \item For all $\alpha<\omega_1$, there is an $\frag$-structure $\mathcal M\models T$, tuple $\V b\in M^{\len(\V k_s)}$, and linear-ordered set $\langle A,<\rangle$ with $|A|=\beth_\alpha$ and $A\subseteq M$ such that 
    \[
      (\mathcal M,{=}/{\sim},\V a/\V c_s, \V b/\V k_s)\models s\text{ for all $\V a\in[A]^{\len(\V c_s)}$}.
    \]
    where we take $(\mathcal M,{=}/{\sim},\V a/\V c_s,\V b/\V k_s)$ to mean the model extending $\mathcal M$ by interpreting the $\sim$ relation as exact equality (1 for equal, 0 for not) and with each element of $\V c_s$ represented by the corresponding element in $\V a$ and similarly for $\V k_s$ and $\V b$.
    \end{itemize}
    We claim that $\mathscr{S}$ is a consistency property.
    The Consistency, Introduction, $\rightarrow$, $\wedge$, $\forall$, and $\Q$ rules are all proven in the straightforward way.
    Proving the $\exists$-rule requires taking some unreferenced constant in $K$ to represent an approximate witness, which is a relatively standard approach (and the only reason for the set of constants $K$).
    We distribute the proofs of the rest of the properties into claims.
    \begin{claim}{1}
      The set $\mathscr{S}$ is non-empty.
      \begin{claimproof}
        Take $s=\{(\forall x)\,x\sim x\}$.
        Clearly $s$ is finite and doesn't reference any of $C$ and $K$, so we've satisfied the first requirement.
        
        Fix any $\alpha<\omega_1$.
        We know by assumption that there is a $\frag$-structure $\mathcal M\models T$ of size $\beth_\alpha$.
        Clearly $(\mathcal M,{=}/{\sim})\models s$.
        As the linear-ordered set is not necessary, we can simply consider any linear-ordering $\langle M,<\rangle$ of $M$ and be done.
        Thus, $s\in \mathscr{S}$, making $\mathscr{S}$ non-empty.
      \end{claimproof}
    \end{claim}
    \begin{claim}{2}
      The set $\mathscr{S}$ satisfies the $\vee$-rule.
      \begin{claimproof}
        Fix a set $s\in \mathscr{S}$ and formula $(\rat q\to\unvee\Phi)\in s$, where $\Phi$ is a countable set of formulas.
        Let $p\in\Q\cap[0,q)$ be given.
        Define $n<\omega$ as the number of constants in $C$ referenced by $s$ (ie.\ $n:=\len(\V c_s)$).
        
        Consider any $\alpha<\omega_1$.
        By definition of $s$ being in $\mathscr{S}$, we know there is a model $\mathcal M$, tuple $\V b\in M^{\len(\V k_s)}$, and linearly ordered set $\langle A,<\rangle$ with $|A|=\beth_{\alpha+n}$ and $A\subseteq M$ such that
        \[
		      (\mathcal M,{=}/{\sim},\V a/\V c_s, \V b/\V k_s)\models s\text{ for all $\V a\in[A]^n$}.
		    \]
		    This means that $\unvee\Phi$ evaluates to at least $q$ in each of those models, so for each $\V a\in[A]^n$, we know we can choose some formula $\phi_{\V a}\in\Phi$ that satisfies
		    \[
		      (\mathcal M,{=}/{\sim},\V a/\V c_s, \V b/\V k_s)\models s\cup\{\rat p\to\phi_{\V a}\}.
		    \]
		    Using the Erd\"os-Rado result (Corollary~\ref{cor:erdos-rado-omega}) with the function from $[A]^n$ to $\Phi$ defined by $\V a\mapsto \phi_{\V a}$, we can take $A_0\subseteq A$ with $|A_0|=\beth_\alpha$ such that $\phi_{\V a}$ is the same sentence for each $\V a\in[A_0]^n$.
		    Denote that sentence $\phi_\alpha$.
		    
		    We can repeat the process above to get a sequence $\{\phi_\alpha\where \alpha<\omega_1\}$.
		    There are only countably many elements of $\Phi$, so we may choose a sentence $\phi$ such that $\phi=\phi_\alpha$ for arbitrarily high $\alpha<\omega_1$.
		    
		    For each $\alpha<\omega_1$ such that $\phi=\phi_\alpha$, we see that there must be a structure with linearly-ordered set of size $\beth_\alpha$ which models $s$ in the desired way.
		    For any $\alpha<\omega_1$ where $\phi\neq\phi_\alpha$, we can take a model with an ordered set larger that $\beth_\alpha$ and simply remove elements until it is exactly of size $\beth_\alpha$.
      \end{claimproof}
    \end{claim}
    \begin{claim}{3}
      The set $\mathscr{S}$ satisfies the Narrowing rule.
      \begin{claimproof}
         Fix a set $s\in \mathscr{S}$ and formula $\phi$ constrained in $s$.
         Let $\epsilon>0$ be given.
         Denote by $\mathcal Q$ the set
         \[
           \mathcal Q:=\{[p,q]\where p,q\in\Q\cap[0,1],\ 0\leq q-p<\epsilon\}
         \]
         Clearly $\mathcal Q$ is countable.
         
         The rest of this claim proceeds the same way as the claim above, except that instead of mapping ordered-tuples into some countable set of formulas, one maps ordered-tuples into $\mathcal Q$.
      \end{claimproof}
    \end{claim}
    
    The above three claims show that $\mathscr{S}$ is a consistency property.
    As stated above, we now wish to find a model of $T\cup I\cup E$ using the extended model existence theorem.
    
    Fix some $s\in \mathscr{S}$, $\phi\in T\cup I\cup E$, and $r\in\Q\cap[0,1)$.
    We wish to show that $s\cup\{\rat r\to\phi\}\in \mathscr{S}$.
    One can show straight from the definition of $\mathscr{S}$ that if $\phi\in T\cup E$, then $s\cup\{\rat 1\to\phi\}\in \mathscr{S}$, which implies $s\cup\{\rat r\to\phi\}\in \mathscr{S}$.
    
    Assume that $\phi\in I$, meaning that $\phi$ is of the form
    \[
      \psi(c_{i_1},\dots,c_{i_n})\leftrightarrow \psi(c_{j_1},\dots,c_{j_n})
    \]
    where $\psi(\V x)$ is a $\frag$-formula.
    Define $\mathcal Q$ as
    \[
      \mathcal Q:=\{[p,q]\where p,q\in\Q\cap[0,1],\ 0\leq q-p<1-r\}.
    \]
    By definition of $s$ being in $\mathscr{S}$, we know there is a model $\mathcal M$, tuple $\V b\in M^{\len(\V k_s)}$, and linearly ordered set $\langle A,<\rangle$ with $|A|=\beth_{\alpha+n}$ and $A\subseteq M$ such that
    \[
		  (\mathcal M,{=}/{\sim},\V a/\V c_s, \V b/\V k_s)\models s\text{ for all $\V a\in[A]^n$}.
		\]
    Using axiom of choice, we can choose a function $f$ from $[A]^n$ into $\mathcal Q$ such that for all $\V a$ we have
    \[
      \psi^\mathcal M(\V a)\in f(\V a).
    \]
    From here, the proof follows much as in claim 2: we use Erd\"os-Rado to find some range containing $\psi^\mathcal(\V a)$ for $\beth_\alpha$ many increasing tuples, then we argue that there is some range appearing for arbitrarily high $\alpha$ and demonstrate that that range holds for all $\alpha$ using DLST.
    
    The important thing to note is that if $\psi^\mathcal M(\V a)$ and $\psi^\mathcal M(\V b)$ are in the same interval in $\mathcal Q$, then:
    \[
      \mathcal M\models \rat r\to(\psi[\V a]\leftrightarrow\psi[\V b]).
    \]
    Therefore, $s\cup\{\rat r\to\phi\}$ is in $\mathscr{S}$. 
    
    That completes all that was necessary to invoke model existence, so we are done.
  \end{proof}
  \begin{corollary}\label{cor:lst}
    If a $\frag$-theory $T$ has a model of size at least $\beth_{\omega_1}$, then $T$ has a model of any infinite cardinality.
    \begin{proof}
      By Downward L\"owenheim-Skolem-Tarski, we know that $T$ has a model of size $\beth_\alpha$ for all $\alpha<\omega$, so we may apply the previous theorem to see $T$ has a model of arbitrary infinite cardinality.
    \end{proof}
  \end{corollary}
\end{theorem}

\section{In Relation to Continuous First-Order Logic}
\newcommand{\Tcont}{{T_\text{cont}}}
\newcommand{\Td}{{T_d}}
This section is devoted to relating the work in this paper to that of continuous first-order logic.
As stated in the introduction, this entire investigation began as an adaptation of model existence to the logic in~\cite{eagle2014omitting}, which is a continuous logic.
We had to step back though as there is no direct way to adapt Skolem functions while maintaining continuity of all functions.

This section shows that we can still get results for continuous logic by selectively requiring continuity.
While simple, this approach appears to be relatively novel in the field.
Specifically, we show that the Hanf number is $\beth_{\omega_1}$.

The continuous logic defined below is the same as for~\cite{eagle2014omitting}, which is an infinitary version of the logic in~\cite{yaacov2008model}.
One of continuous first-order logic's distinguishing features is that it requires structures to be complete relative to their metric.
We do not do so here, though we do focus on results for such structures (which we refer to as \emph{complete} metric structures).

\begin{definition}
  A \emph{continuous signature} is a signature that includes a \emph{modulus of continuity} for each function and relation symbol, meaning that for each function symbol $f$ and relation symbol $R$, there is an increasing function $\delta_f:(0,1]\to(0,1]$ and $\delta_R:(0,1]\to(0,1]$ in $S$.
  The languages $\lang(S)$ and $\baselang(S)$ are defined as before except but with a new binary relation $d$ included.
  A \emph{metric $S$-structure} is a structure $\mathcal M$ such that $d^\mathcal M$ is a metric on $M$.
  A \emph{continuous} metric $S$-structure $\mathcal N$ satisfies each modulus of continuity with respect to $d^\mathcal N$, meaning that for each $\epsilon>0$ and $n$-ary function $f$ and $n$-ary relation $R$ in $S$, the following holds for all $\V a,\V b\in N^n$
  \[
    \text{if $d^\mathcal N(a_i,b_i)<\delta_f(\epsilon)$ for all $i\in\{1,\dots,n\}$, then $d(f^\mathcal N(\V a),f^\mathcal N(\V b))<\epsilon$}
  \]
  and
  \[
    \text{if $d^\mathcal N(a_i,b_i)<\delta_f(\epsilon)$ for all $i\in\{1,\dots,n\}$, then $|R^\mathcal N(\V a)-R^\mathcal N(\V b)|<\epsilon$}.
  \]
  A \emph{pseudometric} structure is the same as a metric structure except that the metric symbol $d$ is interpreted as a pseudometric, meaning that distinct elements may be distance 0 from each other.
  A \emph{complete} metric structure is a structure which is complete relative to its metric.
\end{definition}
\begin{remark}
  For this section, we take $S$ to be a fixed, countable, continuous signature.
\end{remark}
\begin{notation}
  Because we talk about continuous logic in the context of $[0,1]$-valued logic, we refer to structures in $[0,1]$-valued logic as \emph{unrestricted} structures.
  In other words, when using the model existence theorem from the previous section, we are guaranteed an unrestricted structure satisfying a theory (which we then argue is actually a continuous structure).
\end{notation}
\begin{lemma}
  Every formula in finite continuous logic satisfies some derivable modulus of continuity.
  \begin{proof}
    The atomic formulas certainly have moduli.
    The rational connectives are all constant (aka.\ very continuous).
    The $\to$, $\forall$, and finite $\wedge$ all have derivable moduli, though they are generally narrower than their components.
  \end{proof}
\end{lemma}
\begin{remark}
  Having continuity for all formulas is a useful tool.
  Continuous model theorists often hop from pseudometric structures to metric structures and then to complete metric structures entirely because continuity allows them to do so.
  However, even when all atomic formulas are continuous, we get discontinuous formulas in the infinitary logic.
  Recall that Theorem~\ref{thm:expressive-logic} showed there is a formula for every \emph{measurable} function (in particular, the floor function).
  We get the continuity tools back with the following definition.
\end{remark}
\begin{definition}
  A \emph{continuous} fragment is a fragment for which each formula has a modulus of continuity that holds across all metric structures.
\end{definition}
\begin{remark}
  When the atomic formulas all have moduli of continuity, the finite fragment is a continuous fragment.
\end{remark}
\begin{notation}
  For any $n<\omega$ and tuples of terms $\V x$ and $\V y$ of length $n$, we define the formula $d(\V x, \V y)$ to be shorthand for
  \[
    \bigvee_{i=1}^nd(x_i, y_i).
  \]
\end{notation}
\begin{definition}
  We define the theory $T_d$ as the three sentences
  \begin{align*}
    \forall x\,\neg d(x, x),&&
    \forall x\forall y\,d(x, y)\to d(y, x),&&
    \forall x\forall y\forall z\,d(x, z)\to(\neg d(x, y)\to d(y, z)),
  \end{align*}
  which correspond respectively to reflexivity, commutativity, and triangle inequality.
  This definition forces any structure satisfying $T_d$ to be a pseudometric structure.
  
  Define the theory $\Tcont$ as (where $\mathscr F$ and $\mathscr R$ are the function and relation symbols of $S$, respectively)
  \begin{align*}
    T_\text{cont}:=\ &T_d\\
    &\cup\Big\{\Big(\rat p\to d(\V x, \V y)\Big)\vee \Big(d(f(\V x), f(\V y))\to\rat q\Big)\where f\in\mathscr F,\ p,q\in\Q\cap(0,1],\ p<\delta_f(q)\}\\
    &\cup\Big\{\Big(\rat p\to d(\V x, \V y)\Big)\vee \Big(\rat q\to (R(\V x)\to R(\V y))\Big)\where R\in\mathscr R,\ p,q\in\Q\cap(0,1],\ p<\delta_R(q)\}.
  \end{align*}
  We see that the unrestricted structures satisfying $T_\text{cont}$ are precisely those continuous pseudometric structures satisfying the moduli defined in $S$.
\end{definition}
\begin{remark}
  Given the notation of the previous definition, we can redefine a \emph{continuous} metric structure as any metric structure realizing $\Tcont$.
  Similarly for continuous pseudometric structures.
\end{remark}
\begin{remark}
  Pseudometric structures, called \emph{prestructures} in~\cite{yaacov2008model}, appear often when building models.
  They are typically temporary, as shown by the following lemma.
\end{remark}
\begin{lemma}
  For any continuous fragment $\frag$ and continuous pseudometric structure $\mathcal M$, there is a metric structure $\mathcal N$ for which $\mathcal M\equiv_\frag\mathcal N$.
  \begin{proof}
    Take the quotient of $\mathcal M$ over $d^\mathcal M$ in the natural way.
  \end{proof}
\end{lemma}
\begin{definition}
  For any continuous metric structure $\mathcal M$, we define the \emph{completion} of $\mathcal M$, denoted $\overline{\mathcal M}$, as follows:
  \begin{itemize}
  \item
    The universe of $\overline{\mathcal M}$, denoted $\overline M$, is the completion of $M$.
  \item
    The functions of $\overline{\mathcal M}$ are the unique extensions of the functions in $\mathcal M$ to $\overline M$ (unique because each function is continuous and $M$ is dense in $\overline M$).
  \item
    The relations of $\overline{\mathcal M}$ are the unique extensions of the relations in $\mathcal M$ to $\overline M$.
  \end{itemize}
\end{definition}
\begin{lemma}\label{lem:completion}
  For any continuous fragment $\frag$ and continuous metric structure $\mathcal M$, we have $\mathcal M\preceq_\frag\overline{\mathcal M}$.
  \begin{proof}
    The proof is quick from the definitions of $\overline{\mathcal M}$ and $\preceq_\frag$.
  \end{proof}
\end{lemma}
\begin{theorem}[Completeness for Continuous Logic]
  Let $\frag$ be a continuous fragment.
  For any $\frag$-theory $T$ and $\frag$-sentence $\phi$, we have $T\cup\Tcont\proves\phi$ if and only if $\mathcal N\models\phi$ for every complete continuous metric structure $\mathcal N$.
  \begin{proof}
    The forward direction is by soundness.
    
    The other direction is by contrapositive.
    Assume that $T\cup\Tcont\not\proves\phi$.
    By completeness of $[0,1]$-valued logic, we know there is some unrestricted model $\mathcal M\models T\cup\Tcont$ such that $\phi^\mathcal M<1$.
    Because $\mathcal M\models\Tcont$, we know $\mathcal M$ is a continuous pseudometric structure.
    Take $\mathcal N$ to be the quotient structure of $\mathcal M$, so that $\mathcal N\equiv_\frag\mathcal M$.
    We see that $\overline{\mathcal N}$ is a complete metric structure equivalent to $\mathcal N$ on $\frag$, so the value of $\phi^{\overline{\mathcal N}}$ is less than 1.
  \end{proof}
\end{theorem}
\begin{remark}
  Completeness is underwhelming in continuous first-order, but of course one can transfer any of the results from the completeness section to try to get more specific results.
  For instance, one can show Pavelka-style completeness holds for the finite fragment (because the finite fragment is continuous).
  That result was shown more directly by~\cite{yaacov2010proof}, though it is relatively straightforward from fuzzy logic completeness proofs as well.
\end{remark}
\begin{remark}
  We now move onto the Hanf number part of this section.
  It should be stated that, in general, continuous model theorists are not concerned with the size of a complete metric structure.
  As implied by Lemma~\ref{lem:completion}, the behavior of any complete continuous metric structure is fully determined by any continuous substructure whose universe is dense within the complete universe.
  Therefore, the behavior or dense subsets of the universe is in someway more important than the universe itself.
\end{remark}
\begin{definition}
  For any metric structure, we define its \emph{density} to be the smallest cardinality for any dense subset of the structure's universe.
\end{definition}
\begin{theorem}[Hanf Number for Continuous Logic]\label{thm:hanf-continuous}
  Let $\frag$ be a continuous fragment and $T$ a $\frag$-theory.
  If $T$ has continuous metric models of density $\beth_\alpha$ for all infinite $\alpha<\omega_1$, then $T$ has a model with any arbitrary infinite density.
  \begin{proof}
    Let a cardinal $\lambda$ be given; we will show there is a complete continuous metric structure of density $\lambda$.
  
    Recall the notation $\fragsk$ from Definition~\ref{def:skolem}.
    Let $\frag^*$ be the fragment $\frag$ closed under the floor formula.
    
    Using the models and structures of the previous sections, we know that the hypothesis states $T\cup \Tcont$ has a model of arbitrarily high cardinal below $\beth_{\omega_1}$.
    Therefore, $T\cup\Tcont\cup\Tskolem$ has a model for any cardinal below $\beth_{\omega_1}$.
    Because infinitary $[0,1]$-valued logic has Hanf number $\beth_{\omega_1}$, we know there is some structure $\mathcal M$ realizing $T\cup\Tcont\cup\Tskolem$ with an infinite sequence of indiscernibles.
    
    Therefore, by the stretching theorem~(\ref{thm:stretching}), we know there is a $\fragsk^*$-structure $\mathcal M\models T\cup\Tcont\cup\Tskolem$ with a sequence of indiscernibles $\langle A,<\rangle$ of length $\lambda$.
    Recall from the proof of the stretching theorem that $\mathcal M$ is just $\hull(A)$, so $|M|=\lambda$.

    By the definition of indiscernibles, we know that for any $a_1,a_2,a_3,a_4\in A$, the structures $(\mathcal M, a_1, a_2)$ and $(\mathcal M, a_3, a_4)$ are equivalent to each other over $\fragsk^*$.
    Specifically, $d^\mathcal M(a_1,a_2)=d^\mathcal M(a_3,a_4)$.
    This means that $A$ is an equidistantly spaced set in $M$.
    
    Technically, it is possible that $d^\mathcal M(a_1,a_2)$ is just 0, but this can be avoided by modifying the theory $E$ in the proof of Theorem~\ref{thm:hanf} to be
    \[
      E:=\{(\forall x)\neg d(x, x)\}\cup\{\neg\strict{\neg d(c_i, c_j)}\where i\neq j\}.
    \]
    Therefore, we know that all of $A$ is equally (and positively) spaced within $M$.
    Thus, the density of $\mathcal M$ is at least $|A|$, which is $\lambda$.
    Because $|M|=\lambda$, we know that the density of $\mathcal M$ is exactly $\lambda$.
    
    Let $\mathcal N$ be the quotient structure of the reduction of $\mathcal M$ to $\lang(S)$ (the language of $\frag$).
    We know that $\mathcal N$ is a metric structure, but it is also continuous due to being a model of $\Tcont$.
    Therefore, $\overline{\mathcal N}$ is a complete continuous metric model of $T$.
  \end{proof}
\end{theorem}

\section{Morley's Two Cardinal Theorem}
  For this section, we fix a countable language $L$ containing a unary predicate symbol $U$ and we fix a countable fragment $\mathcal L$ of $L_{\omega_1,\omega}$.
  \begin{remark}
    In classical logic, one often thinks of a unary relation as a set, so that for a model $\mathcal M$ and relation $U$, one would interchangeably say $x\in U^\mathcal M$ and $U^\mathcal M(x)$.
    We adopt this notation, so that $U^\mathcal M$ will be thought of as both a real-valued relation and the set $\{x\in M\where U^\mathcal M(x)=1\}$.
  \end{remark}
  \begin{definition}
    (Note should cite Grossberg pg. 194).
    Let $T$ be a first-order $L$-theory.
    For cardinals $\lambda\geq \mu$, we say that $T$ \emph{admits a model of type $\langle\lambda,\mu\rangle$} iff there exists a model $\mathcal M\models T$ of size $\lambda$ such that $|U^\mathcal M|=\mu$.
  \end{definition}
  \begin{definition}
    For any model $\mathcal M$ linear ordered set $\langle A,<\rangle$ with $A\subseteq M$, we say that $\langle A,<\rangle$ is a \emph{sequence of indiscernibles over $U$ in $\frag$} if for all finite tuples $\V u$ of $U^\mathcal M$ and $\V a,\V b\in [A]^{<\omega}$ with $\len(\V a)=\len(\V b)$, we have:
    \[
      (\mathcal M, \V u, \V a)\equiv_\frag (\mathcal M,\V u, \V b)
    \]
  \end{definition}
  \begin{lemma}
    Let $T$ be a $\frag$-theory.
    If for each ordinal $\alpha<\omega_1$ there exists a cardinal $\mu$ such that $T$ admits a model of type $\langle\beth_{\alpha}(\mu),\mu\rangle$, then there is a model $\mathcal M\models T$ with $U^\mathcal M$ being infinite.
    \begin{proof}
      This proof is very similar the one for Theorem~\ref{thm:hanf}. 
      
      Expand the fragment $\frag$ to a countable fragment $\frag^*$ with a new binary relation symbol $\sim$ and two new countable set of constants $K$ and $C=\{c_1,c_2,\dots\}$.
      In addition, take $\frag^*$ to be closed under flooring.
    Define the theory $I$ as the set of all $\frag^*$-sentences of the form
    \[
      \forall u_1\cdots\forall u_m\Big(\psi(u_1,\dots,u_m;c_{i_1},\dots,c_{i_n})\leftrightarrow \psi(u_1,\dots,u_m;c_{j_1},\dots,c_{j_n})\Big)
    \]
    where $n<\omega$, $m<\omega$, $i_1<\cdots<i_n$, $j_1<\cdots<j_n$, and $\psi(\V x;\V y)\in\frag$.
    Define the $\frag^*$-theory $E$ by
    \[
      E:=\{(\forall x)\,x\sim x\}\cup\{\neg(c_i\sim c_j) \mid i\neq j\}
    \]
    One can see that $E$ specifies each of the elements of $C$ are represented by distinct elements and that $I$ stipulates the set $C$ (ordered by index) forms a sequence of indiscernibles.
    
    Take $T'$ to be the $\frag^*$-theory to be $T$ with sentences that stipulate that $U$ is infinitely realized.
    
    Therefore, any model of $T'\cup I\cup E$ reducted to $\frag$ will be a model of $T$ with an infinite set of indiscernibles.
    So we need only find a model of $T'\cup I\cup E$, which we will accomplish with the Extended Model Existence Theorem~(\ref{thm:model-existence-extended}), which requires us to define a consistency property.
    
    Take the set $\mathscr{S}$ to contain of all finite $\frag^*$-constraint-theories $s$ such that:
    \begin{itemize}
    \item Only finitely many of $C$ and $K$ are referenced in $s$, which we denote $\V c_s$ and $\V k_s$ respectively.
    \item For all $\alpha<\omega_1$, there is an $\frag$-structure $\mathcal M\models T$, tuple $\V b\in M^{\len(\V k_s)}$, and linear-ordered set $\langle A,<\rangle$ with $|A|=\beth_\alpha$ and $A\subseteq M$ such that 
    \[
      (\mathcal M,{=}/{\sim},\V a/\V c_s, \V b/\V k_s)\models s\text{ for all $\V a\in[A]^{\len(\V c_s)}$}.
    \]
    where we take $(\mathcal M,{=}/{\sim},\V a/\V c_s,\V b/\V k_s)$ to mean the model extending $\mathcal M$ by interpreting the $\sim$ relation as exact equality (1 for equal, 0 for not) and with each element of $\V c_s$ represented by the corresponding element in $\V a$ and similarly for $\V k_s$ and $\V b$.
    \end{itemize}
    We claim that $\mathscr{S}$ is a consistency property.
    The Consistency, $\rightarrow$, $\wedge$, $\forall$, and $\Q$ rules are all proven in the straightforward way.
    Proving the $\exists$-rule requires taking some unreferenced constant in $K$ to represent an approximate witness, which is relatively standard (and the only reason for the set of constants $K$).
    \end{proof}
  \end{lemma}
  \begin{theorem}
    Let $T$ be a $\frag$-theory.
    If for each ordinal $\alpha<\omega_1$ there exists a cardinal $\mu$ such that $T$ admits a model of type $\langle\beth_{\alpha}(\mu),\mu\rangle$, then for any infinite cardinal $\lambda$, $T$ admits a model of type $\langle\lambda,\aleph_0\rangle$.
    \begin{proof}
      
    \end{proof}
  \end{theorem}
\subsection{Morley's Two Cardinal Theorem}
  For this section, we fix a countable language $L$ containing a unary predicate symbol $U$ and we fix a countable fragment $\mathcal L$ of $L_{\omega_1,\omega}$.
  \begin{remark}
    In classical logic, one often thinks of a unary relation as a set, so that for a model $\mathcal M$ and relation $U$, one would interchangeably say $x\in U^\mathcal M$ and $U^\mathcal M(x)$.
    We adopt this notation, so that $U^\mathcal M$ will be thought of as both a real-valued relation and the set $\{x\in M\where U^\mathcal M(x)=1\}$.
  \end{remark}
  \begin{definition}
    (Note should cite Grossberg pg. 194).
    Let $T$ be a first-order $L$-theory.
    For cardinals $\lambda\geq \mu$, we say that $T$ \emph{admits a model of type $\langle\lambda,\mu\rangle$} iff there exists a model $\mathcal M\models T$ of size $\lambda$ such that $|U^\mathcal M|=\mu$.
  \end{definition}
  \begin{definition}
    For any model $\mathcal M$ linear ordered set $\langle A,<\rangle$ with $A\subseteq M$, we say that $\langle A,<\rangle$ is a \emph{sequence of indiscernibles over $U$ in $\frag$} if for all finite tuples $\V u$ of $U^\mathcal M$ and $\V a,\V b\in [A]^{<\omega}$ with $\len(\V a)=\len(\V b)$, we have:
    \[
      (\mathcal M, \V u, \V a)\equiv_\frag (\mathcal M,\V u, \V b)
    \]
  \end{definition}
  \begin{lemma}
    Let $T$ be a $\frag$-theory.
    If for each ordinal $\alpha<\omega_1$ there exists a cardinal $\mu$ such that $T$ admits a model of type $\langle\beth_{\alpha}(\mu),\mu\rangle$, then there is a model $\mathcal M\models T$ with $U^\mathcal M$ being infinite.
    \begin{proof}
      This proof is very similar the one for Theorem~\ref{thm:hanf}. 
      
      Expand the fragment $\frag$ to a countable fragment $\frag^*$ with a new binary relation symbol $\sim$ and two new countable set of constants $K$ and $C=\{c_1,c_2,\dots\}$.
      In addition, take $\frag^*$ to be closed under flooring.
    Define the theory $I$ as the set of all $\frag^*$-sentences of the form
    \[
      \forall u_1\cdots\forall u_m\Big(\psi(u_1,\dots,u_m;c_{i_1},\dots,c_{i_n})\leftrightarrow \psi(u_1,\dots,u_m;c_{j_1},\dots,c_{j_n})\Big)
    \]
    where $n<\omega$, $m<\omega$, $i_1<\cdots<i_n$, $j_1<\cdots<j_n$, and $\psi(\V x;\V y)\in\frag$.
    Define the $\frag^*$-theory $E$ by
    \[
      E:=\{(\forall x)\,x\sim x\}\cup\{\neg(c_i\sim c_j) \mid i\neq j\}
    \]
    One can see that $E$ specifies each of the elements of $C$ are represented by distinct elements and that $I$ stipulates the set $C$ (ordered by index) forms a sequence of indiscernibles.
    
    Take $T'$ to be the $\frag^*$-theory to be $T$ with sentences that stipulate that $U$ is infinitely realized.
    
    Therefore, any model of $T'\cup I\cup E$ reducted to $\frag$ will be a model of $T$ with an infinite set of indiscernibles.
    So we need only find a model of $T'\cup I\cup E$, which we will accomplish with the Extended Model Existence Theorem~(\ref{thm:model-existence-extended}), which requires us to define a consistency property.
    
    Take the set $\mathscr{S}$ to contain of all finite $\frag^*$-constraint-theories $s$ such that:
    \begin{itemize}
    \item Only finitely many of $C$ and $K$ are referenced in $s$, which we denote $\V c_s$ and $\V k_s$ respectively.
    \item For all $\alpha<\omega_1$, there is an $\frag$-structure $\mathcal M\models T$, tuple $\V b\in M^{\len(\V k_s)}$, and linear-ordered set $\langle A,<\rangle$ with $|A|=\beth_\alpha$ and $A\subseteq M$ such that 
    \[
      (\mathcal M,{=}/{\sim},\V a/\V c_s, \V b/\V k_s)\models s\text{ for all $\V a\in[A]^{\len(\V c_s)}$}.
    \]
    where we take $(\mathcal M,{=}/{\sim},\V a/\V c_s,\V b/\V k_s)$ to mean the model extending $\mathcal M$ by interpreting the $\sim$ relation as exact equality (1 for equal, 0 for not) and with each element of $\V c_s$ represented by the corresponding element in $\V a$ and similarly for $\V k_s$ and $\V b$.
    \end{itemize}
    We claim that $\mathscr{S}$ is a consistency property.
    The Consistency, $\rightarrow$, $\wedge$, $\forall$, and $\Q$ rules are all proven in the straightforward way.
    Proving the $\exists$-rule requires taking some unreferenced constant in $K$ to represent an approximate witness, which is relatively standard (and the only reason for the set of constants $K$).
    \end{proof}
  \end{lemma}
  \begin{theorem}
    Let $T$ be a $\frag$-theory.
    If for each ordinal $\alpha<\omega_1$ there exists a cardinal $\mu$ such that $T$ admits a model of type $\langle\beth_{\alpha}(\mu),\mu\rangle$, then for any infinite cardinal $\lambda$, $T$ admits a model of type $\langle\lambda,\aleph_0\rangle$.
    \begin{proof}
      
    \end{proof}
  \end{theorem}


\subsection{Theorem Thing}
\newcommand{\PCd}{{$\text{PC}_\delta$}}
  \begin{definition}
    A class of $L$-models is called a \PCd-class if there is a language $L^*$ extending $L$ and $L^*$-theory $T$ such that every model in the class is a reduct to $L$ of a $L^*$-model satisfying $T$.
  \end{definition}

  \begin{definition}
    The notion of a model being $\lambda$-saturated for some cardinal $\lambda$ is the same as in classical logic.
  \end{definition}
  
  \begin{theorem}
    Any \PCd-class which is categorical in some uncountable cardinal is also $\beth_{\omega_1\cdot\alpha}$-categorical for any ordinal $\alpha\geq 1$.
    \begin{proof}
      Let $\mathcal K$ be a \PCd-class which is $\mu$-categorical for some $\mu>\aleph_0$.
      Fix $\alpha\geq 1$ and define $\lambda:=\beth_{\omega_1\cdot\alpha}$.
      We need to show $\mathcal K$ is $\lambda$-categorical.
      To do this, we fix $\mathcal M$ and $\mathcal N$ in $\mathcal K$ and show that $\mathcal M\cong\mathcal N$.
      
      By definition of $\mathcal K$ being \PCd, we can take $L^*$ to be a language extending $L$ and a $L^*$-theory $T$ such that $\mathcal K$ is all $L^*$-models of $T$ reducted to the language $L$.
      Therefore, we may choose $\mathcal M^*$ and $\mathcal N^*$ to be $L^*$-models of $T$ which reduct to $\mathcal M$ and $\mathcal N$ respectively.
      
      We first claim that $\mathcal M\equiv\mathcal N$.
      Assume not, then there is some $L$-sentence $\phi$ such that $\phi^\mathcal M\neq\phi^\mathcal N$.
      Because $\lambda\geq\beth_{\omega_1}$, we may apply Corollary~\ref{cor:lst} on $\mathcal M^*$ and $\mathcal N^*$ to get $L^*$-structures $\mathcal M^*_0\preceq_{L^*}\mathcal M^*$ and $\mathcal N^*_0\preceq_{L^*}\mathcal N^*$ of size $\mu$.
      Clearly, both $\mathcal M_0^*$ and $\mathcal N_0^*$ realize $T$, but they disagree on $\phi$ and so their reducts to $L$ are not isomorphic, contradicting the $\mu$-categoricity of $\mathcal K$.
      Consequently, $\mathcal M\equiv\mathcal N$.
      
      We claim that $\mathcal M$ and $\mathcal N$ are saturated.
      We'll focus on $\mathcal M$ as the proof for $\mathcal N$ is the same.
      Assume that $\mathcal M$ is not saturated, meaning that there is a set $A\subseteq M$ with $|A|<\lambda$ and a $\mathcal M$-type $\Sigma(x)$ over $A$ such that $\Sigma(x)$ is not realized in $\mathcal M$.
      
      We intend to encode the formulas of $\Sigma(x)$ into the logic.
      Let $U\subseteq M$ be of the same cardinality as $\Sigma(x)\times A^{<\omega}$ and we define the in-logic unary relation $U:M\to[0,1]$ by:
      \[
        U(u)=\left\{\begin{array}{ll}1&\text{if $u\in U$}\\0&\text{otherwise}\end{array}\right.
      \]
      We associate with each $u\in U$ a formula $\phi_u(x, \V y_u)\in\Sigma(x)$ and a tuple $\V a_u\in A^{\len(\V y_u)}$ surjectively.
      Define the binary relation $S:M^2\to[0,1]$ by:
      \[
        S(u, x):=\left\{\begin{array}{ll}
          \phi^\mathcal M_{u}(x,\V a_{u})&\text{if }u\in U\\
          0&\text{otherwise}
        \end{array}\right.
      \]
      Take $L'$ to be the language $L^*$ expanded with unary relation $U$ and binary relation $S$ and take $\mathcal M'$ to be the $L'$-structure $(\mathcal M^*,U,S)$.
      
      Consider the infinitary sentence $\psi$ defined as:
      \newcommand{\back}{\negthickspace\negthickspace}
      \begin{align*}
        &\forall x\, (Ux\vee\neg {U x})\\
        \wedge\,&\bigwedge_{n<\omega}\forall (u_1\in U)\cdots\forall (u_n\in U)\,\exists x\,\strict{S(u_1,x)\wedge\cdots\wedge S(u_n,x)}\\
        \wedge\,&\neg\,\exists x\,\forall(u\in U)\,\strict{S(u, x)}\\
        \wedge\,&\forall(u\in U)\back\back\bigvee_{\phi\in\text{Sent($L$)}}\back\back\exists y_1\cdots\exists y_{n_\phi}\,\forall x\,\strict{S(u, x)\leftrightarrow\phi(x, y_1,\dots,y_{n_\phi})}\Big)
      \end{align*}
      where in that last line we take $n_\phi$ to be one less than the number of free variables in the corresponding sentence $\phi$.
      
      Clearly $\psi$ holds in $\mathcal M'$, which means that the $L'_{\omega_1,\omega}$-theory $T\cup\{\psi\}$ has a model of type $(\lambda,|U|)$.
      Restricting ourselves to the smallest fragment containing $T$, we can invoke Morley's Two Cardinal Theorem to see that $T\cup\{\psi\}$ admits a model of type $\langle\mu,\aleph_0\rangle$, which when reducted to $L$ will not be $\omega_1$-saturated due to the unreducted model satisfying $\psi$.
      [TODO: lemma parts]
    \end{proof}
  \end{theorem}

\newpage
\bibliography{mybib}{}
\bibliographystyle{alpha}

\end{document}
